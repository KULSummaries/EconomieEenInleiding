\section{De Vraagzijde: De Re\"ele Sfeer}

Dit hoofdstuk behandelt de basis van het Keynesiaans model. Het uitgangspunt hierbij is een stabiel prijspeil. Er is dus geen verschil tussen nominaal BBP ($Y$) en het re\"eel BBP ($Q$).

Vaste prijzen willen zeggen dat $AA$ perfect elastisch is en dit is ook logisch om aan te nemen. Vele bedrijven hebben ``idle capacity'', die ze op de KT snel kunnen aanspreken. We kunnen dus veronderstellen dat het aanbod steeds zal kunnen voldoen aan de vraag op de KT.

Bijgevolg concentreren we ons voornamelijk op $AV$ want deze zal de veranderingen in de economie veroorzaken. We nemen het BBP volgens de bestedingsbenadering:
\begin{equation}
  Q = C + I + G + E - Z
\end{equation}

Stel nu een economie zonder overheid of handel, dan is $AV = C + I$ en $Y_b = Y$ want er is geen overheid om belastingen te heffen. We stellen $C$ dan voor door volgende lineaire combinatie:
\begin{equation}
  C = C_0 + c Y_b
\end{equation}

\subsection{Private Consumptie}

$c$ noemen we de \textit{marginale consumptiequote} (MCQ) en geeft in dit geval weer hoe sterk $C$ verandert als $Y_b$ wijzigt. Het berekenen van een \textit{gemiddelde consumptiequote} is ook mogelijk. Deze geeft de totale consumptie-uitgaven per eenheid beschikbaar inkomen weer. De marginale- en gemiddelde consumptiequote zien er respectievelijk als volgt uit:
\begin{align}
  c &= \frac{\Delta C}{\Delta Y_b}\\
  \frac{C}{Y_b} &= \frac{C_0}{Y_b} + c
\end{align}

De MCQ zou tussen de 0.4 en 0.8 schommelen. Hij ligt waarschijnlijk wel eerder aan de kleine kant door ``habit formation'', en zal nog kleiner zijn door angst-/voorzorgs-sparen in tijden van crisis.

Naast consumeren, kan je je geld ook sparen.
\begin{equation}
  Y_b = C + S
\end{equation}
Na herschrijven vinden we het volgende:
\begin{align*}
  S &= Y_b - C \\
  &= Y_b - (C_0 + cY_b) \\
  &= C_0 + (1-c)Y_b \\
  &= -C_0 + sY_b
\end{align*}
Waarin $s$ de marginale spaarquote is, met $s = 1 - c$. Met andere woorden, alles wat je niet consumeert, spaar je. Sparen is tevens een ``lek'' uit de economie. Met $s$ weten we dus hoe groot dit lek is.

\subsection{Investeringen}
Bij investeringen maken we het onderscheid tussen feitelijke en gewenste investeringen ($I$). De eerste zijn ex post, de andere ex ante. Het verschil zit in de ongewenste voorraadveranderingen.

Inversteringsbeslissingen worden gemaakt op basis van de \textit{netto acuele waarde} (NAW). Deze wordt als volgt berekend:
\begin{equation}
  A_n = \frac{S}{(1+i)^n}
\end{equation}

Hoewel we dus kunnen stellen dat inversteringen be\"invloed worden door de de interestvoet, toekomstige opbrengsten, vertrouwen, enzovoort; stellen we bij het Keynesiaans model dat $I$ exogeen gegeven is ($I = \overline{I}$).

\subsection{De Structurele Vorm van een Economisch Model}
Alle formules die we tot nu toe vonden kunnen we combineren om ons model in een vergelijking uit te schrijven (zie vb in slides). We weten dat de economie in evenwicht is als $AV = AA$ en gebruiken dit als basis.

Een andere mogelijke benadering is het gebruiken van zowel $Y = C + S$ en $ Y = C + I$, wat zoveel wil zeggen als $S + I$.

Eens we dit model opgesteld hebben kunnen we aan \textit{comparatieve statica} doen. We nemen het model op slide 31 als uitgangspunt:
\begin{equation}
  Y = C_0 + cY + \overline{I}
\end{equation}

We kunnen dan eenvoudig berekenen wat het effect is van een verandering in $I$:
\begin{equation}
  \frac{\Delta Y}{\Delta I} = \frac{1}{1-c}
\end{equation}

Zulke analyse zullen we ook voor $C_0$ kunnen uitvoeren en later ook voor andere variabelen als we ons model uitbreiden.

\subsection{Een meer Realistisch Model}
Ons nieuw model ziet er als volgt uit:
\begin{equation}
  Y = C + I + G + E - Z
\end{equation}
Het toevoegen van een overheid zorgt ervoor dat we $G$ toevoegen, waarvan we veronderstellen dat deze endogeen gegeven is. Het zorgt ook voor de invoer van taksen, bijgevolg zal $Y_b = Y - T$. Van deze $T$ veronderstellen we dat hij afhangt van $Y$. Bijgevolg geldt dat $T = T_0 + tY$. $t$ is hierbij de marginale belastingsvoet. Dit is een ander lek uit de economie.

\textbf{Balanced budget multiplier}: als de overheid $G$ en $T$ even veel laat toenemen, zal dit dan nog tot groei leiden? In vb. in slides wel. Je moet de multiplicatoren van beide variabelen optellen en dan zie je dat er een positief effect zal zijn.

\subsection{Endogeniteit van het Overheidssaldo}
$D = G - T$. Als $D> 0$ is er een tekort, als $D < 0 $ is er een overschot. Het begrotingsoverschot hangt dus gedeeltelijk af van de economische groei, aangezien $T$ afhangt van $Y$. Als $Y$ sterk groeit zal het tekort dus kleiner zijn.

\subsection{Impact van Handel op Agregatieve Vraag}
Stel dat we nu ook nog $E$ en $Z$ toevoegen aan ons model. We gaan er van uit dat $E$ exogeen gegeven is. $Z$ zal afhangen van de evolutie van het prijspeil in binnen- en buitenland. Ook wisselkoersen zullen hier een effect hebben. We veronderstellen $Z$ als volgt: $Z = Z_0 + zY$. De netto-uitvoer (of lopende rekening) is dan: $E - Z = E - Z_0 - zY$. $z$ is de marginale uitvoerquote en zegt hoeveel van $Y$ er naar het buitenland lekt.

Ons volledig model ziet er als volgt uit:
\begin{equation}
  Y = \frac{C_0 + \overline{I} + \overline{G} - cT_0 + \overline{E} - Z_0}{1-c(1-t) +z}
\end{equation}

De alternatieve evenwichtsvoorwaarde is waar injecties = lekken:
\begin{equation}
  I + G + E =  S + T + Z
\end{equation}
We gebruiken hiervoor dat $Y = C + I + G + E - Z$ en dat $ Y = C + S + T$.

Tot slot nog twee termen:
\begin{description}
  \item[Automatische stabilisatoren]: in recessie verliezen mensen hun job $\Rightarrow$ ze consumeren minder $\Rightarrow$ zelfversterkend effect. Dankzij sociale zekerheid wordt dit effect afgezwakt omdat mensen kunnen blijven consumeren, ook al is er een recessie.
  \item[Discretionaire maatregelen]: actief beleid voeren zodat economie in recessie zich sneller herstelt. Dit gaat over de multiplicatoren die we net zagen.
\end{description}
