\section{Productie en Kosten}
\label{sec:productieEnKosten}
Hoe kan een producent een bepaalde $q$ produceren tegen een zo laag mogelijke prijs? $\Rightarrow$ welke verdeling van inputs?

\subsection{De Productiefunctie}
=hoeveel je voor elke combinatie vann productiefactoren kan produceren. Voor de eerste keer maken we ook onderscheid tussen beschouwde periode.
\begin{description}
	\item[Korte Termijn ($KT$)] = periode waarin sommige productiefactoren niet/zeer moeilijk/zeer kostelijk aan te passen zijn.
	\item[Lange Termijn ($LT$)] = periode waarin alle productiefactoren relatief eenvoudig aan te passen zijn.
\end{description}
We concentreren on op 2 productiefactoren: kapitaal ($K$) en arbeid ($L$) waarbij in de korte termijn arbeid altijd als variabel en kapitaal als vast beschouwd wordt. We bekomen dus volgende formules:
\begin{align}
	q &= f(K,L)\\
	q^{KT} &= f(L,\overline{K}) = f(L)
\end{align}

Productieniveaus kunnen we voorstellen m.b.v. isokwanten die dezelfde $q$ voor verschillende combinaties van $K$ en $L$ weergeven, of grafiek met 1 constante productiefactor zodat q wel afleesbaar is. $q^{KT}$ kan zo afgebeeld worden aangezien $K$ daar constant is.

\subsubsection{Productiviteit van productiefactoren} 
We introduceren de \textbf{gemiddelde fysische productiviteit} ($GFP$) en de \textbf{marginale fysische productiviteit} ($MFP$), waarbij $GFP_L = \frac{q}{L}$; $GFP_k = \frac{q}{k}$; $MFP_L = \frac{\Delta q}{\Delta L}$ en  $MFP_K = \frac{\Delta q}{\Delta K}$. Het concept van \textbf{variabele meeropbrengsten} zegt dat als \'e\'en productiefactor stijgt en alle andere productiefactoren constant blijven, dat zowel $MFP$ als $GFP$ voor die productiefactor aanvankelijk zullen stijgen tot een maximum en nadien terug zullen dalen.
\begin{figure}[htbp]
	\centering
	\includegraphics[scale=0.4]{Images/white.png}
	\caption{Twee voorbeelden de productiefunctie op korte termijn}
	\label{fig:productiefunctieKorteTermijn}
\end{figure}

In figuur~\ref{fig:productiefunctieKorteTermijn} kan je nu dus zien waarom we aannemen dat de productiefunctie dus niet gewoon concaaf stijgend is. Je kan zien in het andere voorbeeld dat $MFP$ en $GFP$ eerst zal stijgen en dan dalen, waarbij $MFP$ wederom de helling in een bepaald punt is en $GFP$ de voerstraal in een punt. Merk ook op dat $MFP$ de $GFP$ snijdt in het maximum. De logica hierachter is analoog aan het snijden van $GK$ en $MK$ in het minimum van $GK$ als beschreven in sectie~\ref{sssec:kostenfuncties}.

De \textbf{marginale technische substitutievoet} ($MTSV$) geeft weer hoe je de ene productiefactor kan substitueren door de andere. Denk hierbij terug aan de marginale substitutievoet uit sectie~\ref{sssec:deMarginaleSubstitutievoet}.

\begin{figure}[htbp]
	\centering
	\includegraphics[scale=0.4]{Images/white.png}
	\caption{Perfecte substituten en perfecte complementen}
	\label{fig:perfecteSubstitutenEnPerfecteComplementen}
\end{figure}

Figuur~\ref{fig:perfecteSubstitutenEnPerfecteComplementen} geeft de twee speciale gevallen van de $MTSV$ weer. Bij \textbf{perfecte substituten} zijn de isokwanten rechten aangezien de $MSTV$ hetzelfde is over de ganse isokwant. De isokwanten van \textbf{perfecte complementen} zien er uit als rechthoekige hoeken. Dit komt doordat een bepaalde $q$ een bepaalde mix van twee inputs nodig heeft. Stel dat je \'e\'en van de twee inputs verhoogt zal dit geen effect hebben op $q$, je zal enkel overschot van die inputfactor hebben. $\Rightarrow MTSV$ is niet berekenbaar zonder bijkomende informatie. De $MTSV$ is namelijk 0 in het verticale deel, oneindig in het horizontale deel en niet gedefini\"eerd in het hoekpunt (er gaan oneindig veel raaklijnen door). M.a.w. deze inputs zijn niet substitueerbaar.

Er is een \textbf{relatie tussen $MTSV$ en $MFP$} v/d inputs, meer bepaald van de verhouding v/d $MFP$'s. Stel $MFP_L$ groot en $MFP_K$ klein $\Rightarrow |MFSV|$ groot (als je er van uit gaat dat $K$ altijd op de y-as staat) want bij 1 extra arbeid $\Rightarrow q$ stijgt sterk. $MFP_K$ is klein dus we kunnen veel kapitaal opgeven om $q$ constant te houden. Algebra\"ish ziet het er als volgt uit:
\begin{align}
	&\Delta q = MFP_L \times \Delta L + MFP_K \times \Delta K \\
	&\Rightarrow 0 = MFP_L \times \Delta L + MFP_K \times \Delta K \\
	&\Rightarrow MTSV = \frac{\Delta K}{\Delta L} = -\frac{MFP_L}{MFP_K}
\end{align}
$|MTSV|$ daalt dus naarmate $L$ toeneemt (nog steeds in de veronderstelling dat $L$ op de x-as staat).

\subsubsection{Schaalopbrengsten}
Op korte termijn krijgen we het effect van variabele meeropbrengsten, die onder meer voor een dalende $|MTSV|$ zorgen, maar op lange termijn zijn alle inputfactoren variabel. Stel dat we veronderstellen dat alle inputfactoren met eenzelfde $x\%$ stijgen, dan krijgen we volgende situaties:
\begin{itemize}
	\item output neemt toe met meer dan $x$\%$\Rightarrow$ \textbf{toenemende schaalopbrengsten}
	\item output neemt toe met $x$\%$\Rightarrow$ \textbf{constante schaalopbrengsten}
	\item output neemt toe met minder dan $x$\%$\Rightarrow$ \textbf{afnemende schaalopbrengsten}
\end{itemize}

\begin{figure}[htbp]
	\centering
	\includegraphics[scale=0.4]{Images/white.png}
	\caption{Voorbeeld afnemende schaalopbrengsten (slide 18)}
	\label{fig:voorbeeldAfnemendeSchaalopbrengsten}
\end{figure}

Je zou altijd constante schaalopbrengsten verwachten. Stel bvb. dat je een exacte replica van een fabriek op een andere plaats bouwt, dan lijkt het logisch dat de output twee keer zo hoog zal zijn. Toch een aantal redenen voor toenemende schaalopbrengsten:
\begin{itemize}
	\item ondeelbaarheden (grotere machines soms effici\"enter)
	\item specialisatie
	\item fysische wetmatigheden (containerschepen)
\end{itemize}
Zo zijn er ook een aantal redenen voor afnemende schaalopbrengsten:
\begin{itemize}
	\item omgevingsfactoren
	\item organisatorische problemen $\rightarrow$ hoe groter hoe meer moeite met alles te co\"ordineren
	\item fysische wetmatigheden
\end{itemize}

Schaalopbrengsten kunnen afhangen van het niveau van $q$. Vb. toenemende schaalopbrengsten bij een lagere $q$, evoluerend naar afnemende schaalopbrengsten bij een hogere $q$.

\subsubsection{De Cobb-Douglas Productiefunctie}
Deze functie wordt voorgesteld als:
\begin{equation}
	f(L,K) = a L^\alpha K^\beta = q
\end{equation}

Stel nu dat we hier de theorie van schaalopbrengsten op toepassen en alle productiefactoren laten vermenigvuldigen met een factor $\lambda$:
\begin{align}
	f(\lambda L, \lambda K) &= a (\lambda L)^\alpha (\lambda K)^\beta	\\
	&= \lambda^{\alpha + \beta} a L^\alpha K^\beta	\\
	&= \lambda^{\alpha + \beta} q
\end{align}

We onderscheiden de volgende gevallen:
\begin{description}
	\item[$\alpha + \beta < 1$]: afnemende schaalopbrengsten
	\item[$\alpha + \beta = 1$]: constante schaalopbrengsten
	\item[$\alpha + \beta > 1$]: toenemende schaalopbrengsten
\end{description}



\subsection{Kosten op Korte Termijn}
\begin{equation}
	TK = wL + rK
\end{equation}
Maar op $KT$ ligt $K$ vast en we bekomen volgende vergelijking:
\begin{align}
	TK_{KT} =& w L(q) + r\overline{K}
		=& VK_{TK}(q) + FK
\end{align}

Merk op dat FK niet in functie staat van $q$ aangezien het vaste kosten zijn, ze kunnen wel vari\"eren in de tijd.

\subsection{Kosten op Lange Termijn}