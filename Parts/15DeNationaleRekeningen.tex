\section{De Nationale Rekeningen}
\subsection{Hoe Meten we Macro-Economische Schommelingen?}
Voor 1930 geen denkkader en statistieken voor macro-economie. Men merkte wel dat de economie soms beter en dan weer slechter ging. Na 1930 zijn meetinstrumenten ontwikkeld om productie, consumptie en inkomen in kaart te brengen via een soort nationale boekhouding $\Rightarrow$ \textit{de nationale rekeningen}.

We willen hiermee voornamelijk economische groei gaan meten, maar hiervoor moeten we de economie tot een geografisch gebied kunnen afbakenen $\Rightarrow$ BBP, BNP en betalingsbalans.

\subsection{Het Bruto Binnenlands Product (BBP)}
\begin{description}
    \item[BBP]: de totale productie aan finale goederen die op een jaar tijd in een land wordt geproduceerd, gewaardeerd tegen de marktprijs. Bruto verwijst naar het feit dat de bruto waardevermindering van de kapitaalstock (afschrijvingen) niet in rekening gebracht worden.

    \item[BNP]: de totale productie aan finale goederen die op een jaar tijd wordt geproduceerd met behulp van productiefactoren die toebehoren aan een bepaald land.
\end{description}

Er zijn drie benaderingen om het BBP te berekenen:
\begin{description}
    \item[Productiebenaderning]: Men gaat in elke fase van de productie van een finaal goed te toegevoegde waarde ($TW$) berekenen en deze optellen voor de hele keten. Op deze manier kom je dan te weten wat het BBP is (zie vb slide 15).

    \item[Bestedingsbenadering]: De economische kringloop leert ons dat de bestedingen van de gezinnen gelijk zouden moeten zijn aan de totale $TW$ van de bedrijven. Bijgevolg kunnen we het BBP ook als volgt berekenen:
    \begin{equation}
        BBP = \text{consumptie binnenland} + \text{export (consumptie buitenland)} - \text{import}
    \end{equation}

    We kunnen dit ook in symbolen gaan uitdrukken:
    \begin{align*}
        \text{Productie} - \text{Export} + \text{Import} &= \text{binnenlandse vraag}\\
        BBP - E + Z &= \text{Binnenlandse vraag}\\
        &= C + G + I
    \end{align*}
    Waarin $I$ de bruto investeringen zijn, $I - \text{Dep}$ zijn dan de netto investeringen. $C$ is de consumptie door gezinnen en $G$ zijn de overheidsuitgaven. In het algemeen komen we uit dat:
    \begin{equation}
        BBP = C + G + I + E - Z
    \end{equation}

    \item[Inkomensbenadering]: weer uit de economische kringloop weten we dat $TW =$ inkomen. Bijgevolg kunnen we het BBP aan de hand van het inkomen (= lonen + winsten + belastingen) berekenen.
\end{description}

We staan ook even stil bij een probleem dat zich stelt bij het berekenen van het BBP: hoe wordt de productie (vb verkeersveiligheid) van de overheid meegeteld? We kennnen de input wel, maar wat is de $TW$? Deze zal benaderd worden door het inkomen van de ambtenarij, maar zal een onderschatting zijn.



\subsection{Het Nationaal Inkomen}
Geeft weer hoeveel inkomen er beschikbaar is in een land, en moet dus rekening houden met mensen die hier wonen, maar in het buitenland verdienen (loon en dividenden) en wat er wegvloeit naar het buitenland. De klemtoon verhuist dus van grondgebied naar inwoners. Er geldt dat:
\begin{equation}
    BNI = BBP + (FIB_{in} - FIB_{uit}) = BBP + NFIB
\end{equation}
waarbij $FIB$ staat voor factorinkomen en $NFIB$ voor netto-factorinkomens. We kunnen ook het netto nationaal inkomen ($NNI$) berekenen:
\begin{equation}
    NNI = BBP - \text{Dep} = BBP + NFIB - \text{Dep}
\end{equation}

Dit is echter nog niet het netto nationaal beschikbaar inkomen ($NNBI$). Hiervoor moeten nog de transferten uit het buitenland in rekening gebracht worden:
\begin{align*}
    NNBI &= BBP + NFIB - \text{Dep} + NTRA \\
        &= C + G + I + E - Z + NFIB - \text{Dep} + NTRA \\
        &= (C + G + I_{netto}) + (E - Z + NFIB + NTRA)
\end{align*}
waarbij dat tweede deel tussen haakjes de lopende rekeningen zijn. Het eerste deel tussen haakjes zijn de binnenlandse bestedingen.

\subsection{De Betalingsbalans}
De betalingsbalans is een soort boekhouding, met onder andere de lopende rekeningen (zie overzicht slide 34).
