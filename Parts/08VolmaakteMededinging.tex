\section{Volmaakte Mededinging}
De volgende hoofdstukken zullen bepaalde marktvormen beschrijven. De marktvorm is namelijk belangrijk in het bepalen van het gedrag van ondernemingen in deze markt. In dit hoofdstuk komt volmaakte mededinging aan bod. De kenmerken hiervan zijn:
\begin{description}
   \item[Prijsnemerschap]: voor elke individuele onderneming geldt dat ze prijsnemer is. Ze produceren allemaal homogene goederen.
   \item[Marktatomisme]: alle ondernemingen klein $\Rightarrow$ strategie niet bepaald door wat anderen doen. Hangt samen met prijsnemerschap, niemand heeft invloed op de markt
   \item[Perfecte informatie]: = er zijn geen zoekkosten. Iedereen weet wat de kwaliteit van het aangeboden product is en bij welke aanbieder ze de laagste prijs kunnen vinden.
   \item[Vrije uit- en toetreding]: geen technologische kennis die maar bij enkele bedrijven gekend is, afwezigheid van wettelijke belemmeringen.
\end{description}
$\Rightarrow$ dit is een onrealistische vorm, sommige markten neigen er wel naar (bvb. aandelenmarkt). Deze marktvorm is wel het beste van uit het welvaartstandpunt. In het algemeen zal men hier de grootste welvaart bereiken.

\begin{figure}[htbp]
   \centering
   \includegraphics[scale=0.4]{Images/white.png}
   \caption{Globale markt en aanbod van \'e\'en producent}
   \label{fig:globaleMarktEnAanbodEenProducent}
\end{figure}

In figuur~\ref{fig:globaleMarktEnAanbodEenProducent} zie je hoe de globale markt er uitziet en het aanbod voor een enkele producent. Dat de globale markt een normale vraag is vrij logisch, de perfect elastische aanbodscurve per producent iets minder. Dit komt echter door het prijsnemerschap. Een producent kan verkopen zoveel hij wil, aan een gegeven prijs. Wordt zijn prijs iets hoger, dan zullen alle klanten bij de concurrentie kopen (perfecte informatie). Daarnaast heeft de producent ook geen invloed op $p$ met de $q$ die hij produceert.

Uit het feit dat $\overline{p}$ volgt dat:
\begin{align}
   TO &= q \times \overline{p} \\
   GO &= \frac{TO(q)}{q} = \frac{q \times \overline{p}}{q} = \overline{p} \\
   MO &= \frac{\Delta TO(q)}{\Delta q} = \frac{\overline{p} \times \Delta q}{\Delta q} = \overline{p}
\end{align}
$TO$ is dus een stijgende rechten, wat er voor zorgt dat $MO$ en $GO$ constanten zijn, en ook aan elkaar gelijk zijn.
