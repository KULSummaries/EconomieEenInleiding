\section{Geld en het Bankwezen}
\label{sec:Geld en het Bankwezen}

\subsection{Geschiedenis van Geld}
\label{sub:Geschiedenis van Geld}
\begin{enumerate}
  \item Ruilhandel: goederen en diensten rechtstreeks uitgewisseld (moeilijk op te potten en groot aantal ruilverhoudingen).
  \item Goederengeld: waardevaste goederen, door iedereen gewild, zijn betaalmiddel (vb schelpen of zout).
  \item Goud en andere metalen: risicovol om te transporteren, niet altijd grantie op gewicht en kwaliteit
  \item Bankbiljetten: schuldbekentenis van goudsmid, kan echter niet gebruikt worden als je verder weg gaat (goudsmid onbekend).
\end{enumerate}

\subsection{Waarom Gebruiken we Geld?}
\label{sub:Waarom Gebruiken we Geld?}
Drie voorname functies:
\begin{itemize}
  \item Waardemeter: gemakkelijk om waardes van verschillende soorten goederen te vergelijken.
  \item Ruilmiddel: handig om te ruilen en geen zoekkosten zoals bij directe ruil
  \item Beleggingsmiddel: laat makkelijk toe koopkracht over te dragen naar volgende periode, je kan ook snel wisselen tegen andere soorten goederen.
\end{itemize}

Geld is echter ook niets waard en is wellicht te gemakkelijk om op te potten en te produceren. Geld voldoet dus aan volgende voorwaarden:
\begin{itemize}
  \item moeilijk namaakbaar
  \item groote waarde per eenheid gewicht (zodat het makkelijk transporteerbaar is)
  \item duurzaam
  \item deelbaar zonder waardeverlies
\end{itemize}
Deze laatste twee zijn eigenschappen van metalen. Ze gaan lang mee, en als je ze in twee kapt, zijn de twee helften samen nog steeds even veel waard.


\subsection{Soorten Geld}
\label{sub:Soorten Geld}

\begin{itemize}
  \item Nationaal geld:
  \begin{itemize}
    \item Chartaal geld ($CP$)
    \item giraal geld (deposito's) ($D$)
  \end{itemize}
  Deze twee vormen samen de \textit{geldhoeveelheid in enge zin} ($M_1$). Tel hierbij de spaarrekeningen en termijdeposito's $\le$ 2 jaar (=quasi geld) en dan bekom je de \textit{geldhoeveelheid in ruime zin} ($M_3$).
  \item Internationaal geld
\end{itemize}

\subsection{Geldcreatie}
\label{sub:Geldcreatie}
Vb van de goudsmid:
\begin{enumerate}
  \item goudsmid bewaart geld van mensen, maar geeft in de plaats schuldbekentenissen waar ze mee kunnen betalen.
  \item goudsmid leent geld uit aan andere mensen, afhankelijk van $r$.
  \item de persoon die geld leent van de goudsmid kan dit op zijn beurt weer deponeren bij een goudsmid.
  \item zie stap 1
\end{enumerate}

Dit systeem berust dus op vertrouwen dat ``klanten'' van de goudsmid hun geld zullen terugkrijgen $\Rightarrow$ \textit{fiduciair geld}.

Voordat we overgaan naar de geldbasismultiplicator, eerst 2 concepten:
\begin{description}
  \item[Geldhoeveelheid] = $M = CP + D$
  \item[Geldbasis] = $MB = CP + R$
\end{description}

De \textit{geldbasismultiplicator} ($mm$) is d everhouding tussen totale geldhoeveelheid ($M$) en de geldbasis ($MB$). Wiskundig gezien:
\begin{align*}
  mm &= \frac{M}{MB} = \frac{CP + D}{CP + R}\\
  mm &= \frac{(CP / D) + 1}{(CP / D) + r} = \frac{c + 1}{c + r}
\end{align*}

Waarbij we gebruiken dat $R/D = r$ en $CP/D = c$ = chartaal geld dat mensen op zak willen houden in verhouding tot deposito's.

\subsection{Het Geldaanbod}
\label{sub:Het Geldaanbod}
We kunnen de formule van de geldbasismultiplicator zeer eenvoudig omvormen tot:
\begin{equation}
  M = mm(c,r) \times MB
\end{equation}
$mm$ is dus een functie van $c$ en $r$, of anders gezegd: van het gedrag van het publiek en van de banken. $MB$ kan dan beschouwd worden als het gedrag van de centrale bank. Volgende zaken leiden allemaal tot een grotere geldcreatie/geldhoeveelheid:

\begin{itemize}
  \item consumenten willen weinig cash geld bijhouden (lage $c$)
  \item banken houden weinig kasreserve aan (lage $r$)
  \item centrale banken brengen veel geld uit (hoge $MB$)
\end{itemize}

De centrale banken hebben dus een effect op de geldhoeveelheid. Dit kan via een directe methode (vb \textit{quantitative easing}), of via de secundaire markt door leningen te verschaffen, het opleggen van een bepaalde $r$ en de rol van \textit{lender of last resort} op zich te nemen.

De stabiliteit van het bankwezen in dit verhaal is ook belangrijk. Er is namelijk een structureel probleem: ``\textit{borrow short and lend long}''. Anderzijds is er altijd de mogelijkheid op een domino-effect wanneer het vertrouwen in het systeem laag is. Dit komt door de imperfecte informatie bij het publiek en het feit dat banken onderling geld aan elkaar lenen.

Vervolgens is er het risico op \textit{moral hazard}, het ``too big to fail'' verhaal. (Er werd ook iets over zeepbellen gezegd.) Is het echter gerechtvaardigd dat de overheid tussen komt in het bankwezen? Volgens boek ja, want bij faillissement:
\begin{itemize}
  \item verliezen mensen hun vermogen
  \item komt kredietverlening tot stilstand
  \item ontstaat er achteruitgang van economische activiteit
\end{itemize}


\subsection{De Geldvraag}
\label{sub:De Geldvraag}
Er zijn verschillende componenten in de vraag:
\begin{itemize}
  \item je hebt geld nodig om transacties te verrichten $\Rightarrow$ \textit{transactievraag naar geld}
  \item wanneer de prijzen stijgen (inflatie) bij gegeven $Q$ heb je meer geld nodig. Ook prijs heeft dus impact op $M^V$. Anderzijds wil je minder cash geld aanhouden bij hoge inflatie.
  \item rendement op lange termijn van beleggingen: hoe hoger, hoe minder geld mensen willen bijhouden op rekeningen $\Rightarrow$ \textit{vermogensvraag naar geld}
\end{itemize}

De transactievraag kennen we via het BBP, de prijs/inflatie via de BBP deflator en de vermogensvraag via de interestvoet. Via Fischer kunnen we dus d egeldvraag bepalen:
\begin{align}
  &M \times V = P \times Q \\
  \Leftrightarrow &M^V = \frac{1}{V} \times P \times Q = \kappa \times P \times Q
\end{align}

We weten dat deze $\kappa$ een functie is van de interestvoet en de verwachte inflatie $\Rightarrow \kappa(\overline{i}, \overline{\pi^e})$. Vervolgens kunnen we net zoals bij het BBP kijken naar de re\"ele geldvraag vs de nominale geldvraag. Deze laatste is:
\begin{align}
  \frac{M^V}{P} &= \kappa(\overline{i}, \overline{\pi^e}) \times Q \\
  &= L_0 + \beta \times Q - \delta \times i \\
  \Leftrightarrow M^V &= P (L_0 + \beta \times Q - \delta \times i)
\end{align}

We onderscheiden dus volgende co\"effici\"enten in de geldvraag: $\beta$ is de inkomensgevoeligheid van de re\"ele geldvraag, $\delta$ is de interestgevoeligheid van de re\"ele geldvraag en $L_0$ is de invloed van alle andere factoren.

\subsection{Evenwicht tussen Geldvraag en Geldaanbod}
\label{sub:Evenwicht tussen Geldvraag en Geldaanbod}
We kunnen het aanbod en de vraag naar geld samenbrengen om zo het evenwicht op de geldmarkt te vinden. Het resultaat zie je in figuur~\ref{fig:geldmarktevenwicht}.
\begin{figure}[htbp]
  \centering
  \includegraphics[scale=0.4]{Images/white.png}
  \caption{Het Geldmarktevenwicht (18.4)}
  \label{fig:geldmarktevenwicht}
\end{figure}
Je kan zien dat we veronderstellen dat het geldaanbod perfect inelastisch is (de centrale bank zet het aanbod).

\subsection{Taylor regel}
\label{sub:Taylor regel}
zie slides \& boek
