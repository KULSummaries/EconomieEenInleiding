\section{Vraag en Aanbod}
We zoeken een mechanisme om beslissingen van econ. agenten samen te brengen\\
$\rightarrow$\textbf{markt-/prijs-mechanisme}

In dit mechanisme gaat het steeds om ruiltransacties tussen twee partijen, gedreven door eigenbelang. De prijsvorming gebeurt door het tot stand komen van een marktevenwicht, via verschuivingen van vraag ($V$) en aanbod ($A$). $V$ en $A$ geven een antwoord op voor wie er wordt geproduceerd, door wie en tegen welke $p$.

\subsection{De Vraag}
Geeft de relatie tussen de gevraagde hoeveelheid ($q^V$) van een bepaald goed en een reeks ``determinanten''. Dit wordt weergeven in een \textbf{algemene vraagfunctie}.
\begin{equation}
	q^{V}_{broodje} = f(p,p_{friet},p_{kebab},p_{pizza},y,seizoen,reclame,...)
\end{equation}

Vaak gaan we echter stapsgewijs te werk en gaan we slechts een variabele aanpassen en het effect daarvan bestuderen, de anderen blijven constant (\textbf{Ceteris Paribus}). Dit is dan een \textbf{parti\"{e}le vraagfunctie}.
\begin{equation}
	q^{V}_{broodje} = f(p | p_{friet},p_{kebab},p_{pizza},y,seizoen,reclame,...)
\end{equation}

Meestal zullen we, zoals bovenstaand voorbeeld, de parti\"{e}le vraagfunctie van de prijs beschouwen. ($p$ is de enige variabele)

\begin{description}
	\item[reservatieprijs] reflecteert de marginale maximale bereidheid tot betalen. In andere woorden: als een consument reeds een bepaalde $q$ geconsumeerd heeft, hoeveel wil hij nog betalen voor een extra eenheid.
\end{description}
\todo[inline]{Voorbeeldtabel reservatieprijzen invoegen}

Consumenten gaan dus consumeren wanneer $p_{markt} < reservatieprijs$, en speelt dus ook een rol in de hoeveelheid die je koopt. Aangezien mensen verschillende reservatieprijzen hebben, zal wanneer $p$ daalt, de $q^V$ stijgen.

$\rightarrow$ dalende vraagfunctie, zie figuur~\ref{fig:vraagfunc}. De oppervlakte onder deze functie is wat een bepaalde $q$ waard is voor de consumenten $\rightarrow$ reflecteert de bereidheid tot betalen.
\begin{figure}[htbp]
	\centering
	\includegraphics[scale=0.4]{Images/white.png}
	\caption{Vraagfunctie met consumentensurplus}
	\label{fig:vraagfunc}
\end{figure}

Waneer we ook een marktprijs introduceren krijgen we het consumentensurplus + wat er effectief betaalt wordt door de consumenten. 

Vraagcurve op 2 manieren leesbaar:
\begin{itemize}
	\item $p \rightarrow q$: hoeveel gekocht onder welke prijs
    \item $q \rightarrow p$: bij bepaalde $q$, wat is men bereid te betalen voor een extra eenheid.
\end{itemize}

\subsection{Het Aanbod}
Volledig analoog aan de vraagcurve. Ook algemene en parti\"{e}le aanbodsfunctie. De reservatieprijs in het geval v/d producenten geeft de marginale kosten weer. Stel er wordt een extra eenheid gevraagd, wat willen ze hier minimaal voor krijgen. In andere woorden: hoeveel zijn de kosten die de prod. van deze extra eenheid met zich meebrengt?

Het producentensurplus is in dit geval de winst, de opp. onder de aanbodcurve de kosten voor die productie. Het resultaat is een aanbodfunctie die er uit ziet als figuur~\ref{fig:aanbodfunc}.
\begin{figure}[htbp]
	\centering
	\includegraphics[scale=0.4]{Images/white.png}
	\caption{Vraagfunctie met producentensurplus}
	\label{fig:aanbodfunc}
\end{figure}


\subsection{Prijsvorming}
\subsubsection{Marktevenwicht}
Vraag en aanbod gaan we samenbrengen zoals in figuur\ref{fig:vraagenaanbod}. $p$ gaat evolueren naar de evenwichtsprijs. Prijzen die te hoog of te laag zijn gaan resp. voor een neerwaartse of opwaartse druk op de prijs zorgen. In het geval van een te lage prijs zijn er nog mensen over die meer willen betalen dan de marktprijs, en er zijn ook aanbieders die voor deze hogere prijs een extra eenheid willen produceren $\rightarrow$ opwaartse druk.
\begin{figure}[htbp]
	\centering
	\includegraphics[scale=0.4]{Images/white.png}
	\caption{Vraag en aanbod}
	\label{fig:vraagenaanbod}
\end{figure}

\subsubsection{Lineaire vraag- en aanbodfunctie}
We gaan $V$ en $A$ voorstellen door rechten, waarbij het marktevenwicht het snijpunt van deze twee rechten is.

\subsubsection{Verschuivingen van vraag- en aanbodcurve}
Veranderingen in andere variabelen dan degene die we constant houden (normaal gezien $p$), zorgen voor verschuiving v/d curve. De rico blijft wel behouden. Deze verschuivingen zullen meestal een nieuwe marktprijs en gevraagde hoeveelheid met zich meebrengen







