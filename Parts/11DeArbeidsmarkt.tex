\section{De Arbeidsmarkt}
\subsection{Loonconcepten}
\begin{description}
   \item[Brutoloon ($w^b$)] = ``marktprijs''
   \item[Nettoloon ($w^n$)] = loon dat de WN overhoudt na belastingen = producentenprijs (WN = producent)
   \item[Loonkosten ($w^g$)] = wat WG effectief betaalt = consumentenprijs
   \item[Loonwig] = loonkosten - nettoloon
\end{description}

Er wordt ook nog het onderscheid gemaakt tussen \textbf{nominaal} en \textbf{re\"eel loon}. Het eerste is wat je krijgt, uitgedrukt in een munteenheid. Het re\"ele loon drukt de koopkracht uit, in goederen, en is gelijk aan $\frac{w}{p}$, waarbij $p$ vaak een \textbf{prijsindex} is.

\subsection{De Vraag naar Arbeid}

We gaan er eerst van uit dat de arbeidsmarkt een markt van perfecte concurrentie is. Op korte termijn weten we dat het volgende geldt:
\begin{equation}
   q^{KT} = f(L,\overline{K}) = f(L)
\end{equation}

Een onderneming wil haar winst maximaliseren. Als we dit niveau kennen weten we hoeveel werknemers er nodig zijn, want $K$ is vast. We zoeken dus waar $MO(q^*) = MK(q^*)$ met $MO$ en $MK$ als volgt:
\begin{align}
   MO(q) &= p \text{ (volmaakte mededinging)}\\
   MK(q) &= \frac{\Delta VK}{\Delta q} = \frac{\Delta (w \times L(q))}{\Delta q} = w \frac{\Delta L(q)}{\Delta q} = \frac{w}{MFP_L}
\end{align}

Hieruit volgt dat het evenwicht zal liggen in $p = \frac{w}{MFP_L}$. We kunnen nu het loon, de prijs op de arbeidsmarkt, als volgt uitdrukken:
\begin{equation}
   w = p \times MFP_L = MWP_L
\end{equation}
waarbij $MWP_L$ de \textbf{Marginale Waardeproductiviteit van Arbeid} is. Het loon zal dus gelijk zijn aan de waarde van de output die \'e\'en extra eenheid arbeid produceert. Als we $w$ in functie van $L$ (de $L$ die de winst optimaliseert) uitzetten in een grafiek, bekomen we figuur~\ref{fig:vraagNaarArbeid}. Deze figuur is niets anders dan de vraagcurve van arbeid.

\begin{figure}[htbp]
   \centering
   \includegraphics[scale=0.4]{Images/white.png}
   \caption{Vraag naar arbeid (slide 9)}
   \label{fig:vraagNaarArbeid}
\end{figure}
