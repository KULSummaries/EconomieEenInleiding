\section{Elasticiteiten en Schokken}
We willen onderzoeken wat er gebeurt wanneer de markt in evenwicht is en er ``iets gebeurt'' (een schok). $\rightarrow$ wat zijn de gevolgen? Anderzijds willen we ook weten wat de gevolgen zijn van overheidsingrijpen.

\subsection{Elasticiteiten}
\subsubsection{Eigen Prijselasticiteit van de Vraag}
\label{sssec:PrijselastVanDeVraag}

= hoe sterk reageert $q^V$ als de prijs wijzigt?

Een voor de hand liggende maatstaf is de rico v/d vraagfunc., maar: een toename v/d prijs voor goed $x$ is niet vergelijkbaar met de prijs v/e ander goed. $\rightarrow$ naar procentuele verandering van $p$ en $q$ kijken.

\begin{equation}
	\varepsilon^{v}_{p} =
    \cfrac{\% \Delta q^V}{\% \Delta p} =
    \cfrac{\cfrac{q^{V}_{1} - q^{V}_{0}}{q_{0}^{V}}}{\cfrac{p_1 - p_0}{p_0}} =
    \cfrac{\cfrac{\Delta q^V}{q_{0}^{V}}}{\cfrac{\Delta p}{p_0}} =
    \cfrac{\Delta q^V}{\Delta p} \times \cfrac{p_0}{q_0}
    \label{eq:prijselasticiteitvandevraag}
\end{equation}


Dankzij logisch nadenken kunnen we verwachten dat $\varepsilon^{v}_{p} < 0$. Daarnaast is de prijselasticiteit niet gelijk aan de rico en ook niet constant. Ze vari\"{e}ert langsheen de vraagrechte. Ze hangt wel samen met de rico.
\begin{figure}[htbp]
	\centering
	\includegraphics[scale=0.4]{Images/white.png}
	\caption{Eigen prijselasticiteit van de vraag (slide 6)}
	\label{fig:eigenPrijselasticiteitVanDeVraag}
\end{figure}

Men maakt het onderscheid tussen \textbf{boogelasticiteit} en \textbf{puntelasticiteit}. Bij die eerste gaat men de elasticiteit tussen twee punten op de rechte berekenen, bij de tweede gaat men via limieten trachten de elasticiteit in een gegeven punt te berekenen. Er zijn drie punten waarbij de elasticiteit een speciale waarde aanneemt:
\begin{description}
	\item[$\varepsilon^{v}_{p} = 0$] In het punt waar $p=0$, en dus waar de vraagrechte de $p$-as snijdt.
    \item[$\varepsilon^{v}_{p} = -\infty$] In het punt waar $q=0$, en dus waar vraagrechte de $q$-as snijdt.
    \item[$\varepsilon^{v}_{p} = -1$] In het punt in het midden van de twee voorgaande punten.
\end{description}

Waarom $-1$ in het midden? We kunnen de vraagrechte voorstellen door $q= \alpha - \beta p$. Dit wil zeggen dat:

\begin{equation}
	\varepsilon^{v}_{p} =
    \cfrac{\Delta q}{\Delta p} \times \cfrac{p}{q} =
    -\beta \times \cfrac{p}{q} =
    \cfrac{-\beta p}{\alpha - \beta p}
\end{equation}

Als we dan ook weten dat $\varepsilon^{v}_{p} = -1$, volgt daaruit dat:

\begin{equation}
	p = \cfrac{\alpha}{2 \beta}
\end{equation}

Wat zich halverwege de vraagfunctie bevindt. \todo[inline]{Voeg plaats in voor tekening nota's}

Er zijn ook twee gevallen waarin $\varepsilon^{v}_{p}$ een extreme waarde aanneemt:
\begin{description}
	\item[Vraagrechte is horizontale lijn] $\varepsilon^{v}_{p} = -\infty$. Perfect prijselastisch, de vraag ``verdampt'' bij prijsveranderingen.
    \item[Vraagrechte is verticale lijn] $\varepsilon^{v}_{p} = 0$ Perfect prijsinelastisch. Wat de prijs ook is, de gevraagde hoeveelheid ($q^V$) blijft altijd constant. Een voorbeeld hiervan is de vraag naar water of andere levensnoodzakelijke goederen.
\end{description}

Elasticiteiten worden dus gebruikt om te kijken wat er zal gebeuren voordat je een prijsstijging doorvoert. $q$ zal en $p$ stijgen, maar wat zal er gebeuren met de totale inkomsten? Zoals je in figuur~\ref{fig:prijsveranderingelasticiteiten} kan zien zal dit van de prijselasticiteit van de vraag afhangen.
\begin{figure}[htbp]
	\centering
	\includegraphics[scale=0.4]{Images/white.png}
	\caption{Verandering van $TO$ bij verandering van de prijs}
	\label{fig:prijsveranderingelasticiteiten}
\end{figure}

$\varepsilon^{v}_{p}$ is groter naarmate:
\begin{itemize}
	\item meer substituten aanwezig zijn (gemakkelijk omschakelen als $p$ stijgt)
    \item vraag minder dringend is. Vb. vraag voedsel $><$ vraag bioscooptickets. Vraag naar voedsel is dringender en een stijging van $p$ gaat minder invloed hebben dan wanneer dit voor bioscooptickets gebeurt.
    \item het bestedingsaandeel voor dat goed groter is.
    \item de beschouwde tijdsperiode langer is. Op korte termijn reageren consumenten matig op veranderingen. Het heeft tijd nodig voordat mensen volledig overschakelen op iets anders.
\end{itemize}



\subsubsection{Inkomenselasticiteit van de Vraag}
Genoteerd als $\varepsilon^{v}_{y}$, en geeft weer in welke mate een verandering van het inkomen een verandering van $q^V$ met zich meebrengt. Onze intu\"itie zegt dat dit verband recht evenredig moet zijn, en dat bijgevolg $\varepsilon^{v}_{y}$ positief zal zijn. De formule is analoog aan formule~\ref{eq:prijselasticiteitvandevraag}, maar $p$ wordt door $y$ vervangen.

Wat we echter zien is dat er een aantal verschillende situaties optreden, met verschillende waarden voor $\varepsilon^{v}_{y}$.
\begin{description}
	\item[$\varepsilon^{v}_{y} > 0$] voor een \textbf{normaal goed}, zoals onze intu\"itie reeds deed aanvoelen.
    \item[$0 < \varepsilon^{v}_{y} < 1$] voor een \textbf{noodzakelijk goed} zoals bvb. eten. Meer inkomen doet de uitgaven aan dit soort goederen stijgen, maar minder $q^V$ stijgt minder hard dan $y$.
    \item[$\varepsilon^{v}_{y} > 1$] voor een \textbf{luxegoed}. Indien $y$ stijgt zal $q^V$ naar dit goed meer dan evenredig stijgen.
    \item[$\varepsilon^{v}_{y} < 0$] voor een \textbf{inferieur goed}. Klompen zijn hier een voorbeeld van. Als $y$ stijgt zal men gewoon betere schoenen willen kopen en $q^V$ daalt bijgevolg.
\end{description}

Je kan ook aan de budgetaandelen zien in welke categorie een goed zich bevind aan de hand van wat hoe het budgetaandeel veranderd als $y$ veranderd.

\subsubsection{Kruiselingse Prijselasticiteit van de Vraag}
$\varepsilon^{v}_{x,z}$ geeft weer wat de vraag naar goed $x$ doet als de prijs van goed $z$ wijzigt. Wederom analoog aan formule~\ref{eq:prijselasticiteitvandevraag} komen we uit dat:

\begin{equation}
	\varepsilon^{v}_{x,z} =
    \cfrac{\Delta q^{V}_{x}}{\Delta p_z} \times \cfrac{p_{z0}}{q_{x0}^{V}}
\end{equation}

A.d.h.v. $\varepsilon^{v}_{x,z}$ classificeren we goederen in drie klassen:
\begin{description}
	\item[substituten] als $\varepsilon^{v}_{x,z} > 0$. Een prijsstijging van het ene goed zal een stijging in de vraag naar het andere goed veroorzaken
    \item[complementen] als $\varepsilon^{v}_{x,z} < 0$. Een prijsstijging van het ene goed zal een daling van de vraag naar het andere goed veroorzaken.
    \item[onafhankelijke goederen] als $\varepsilon^{v}_{x,z} = 0$. De prijs van het een goed heeft geen invloed op de gevraagde hoeveelheid van het andere goed.
\end{description}

\subsubsection{Prijselasticiteit van het Aanbod}
Net zoals bij de prijselasticiteit van de vraag(sectie~\ref{sssec:PrijselastVanDeVraag} maken we nu het onderscheid tussen boog- en puntelasticiteit, waarbij de boogelasticiteit gegeven wordt door:

\begin{equation}
	\varepsilon^{A}_{p} =
    \cfrac{\Delta q^A}{\Delta p} \times \cfrac{p_0}{q_{0}^{A}}
\end{equation}

In tegenstelling tot de prijselasticiteit van de vraag is $\varepsilon^{A}_{p} = +\infty$ wanneer het aanbod perfect elastisch is (horizontale curve) en $\varepsilon^{A}_{p} = 0$ wanneer het aanbod perfect prijsinelastisch is (verticale curve).


\subsection{De Markt in Werking}
\subsubsection{Aanbodschokken}
\begin{figure}[htbp]
	\centering
	\includegraphics[scale=0.4]{Images/white.png}
	\caption{Aanbodschok onder verschillende vraagcurves (slide 26)}
	\label{fig:aanbodschok}
\end{figure}

\begin{description}
	\item[met $V_B$] $p$ stijgt en $q$ daalt als $A$ naar links/boven beweegt. $A$ verschuift over de zelfde afstand als dat de kosten voor de prod. van een extra eenheid voor een gegeven q gestegen zijn.\\
    $\rightarrow$ $p$ en $q$ veranderen in tegengestelde richting, de grootte van verandering afhankelijk v/d elasticiteit v/d vraag.
    \item[met $V_C$] perfect inelastische vraag: $q$ blijft altijd constant en de toename v/d kosten wordt volledig doorgespeeld aan de consument. ($p$ stijgt met zelfde hoeveelheid als de kosten voor de prod. toenemen)
    \item[met $V_D$] perfect prijselastische vraag: $p$ blijft constant en $q$ daalt, maar ook de extra kosten komen nu volledig op de schouders v/d producenten.
\end{description}

\subsubsection{Vraagschokken}
\begin{figure}[htbp]
	\centering
	\includegraphics[scale=0.4]{Images/white.png}
	\caption{Vergelijking van twee markten}
	\label{fig:vergelijkingTweeMarkten}
\end{figure}
Waarom zijn prijsfluctuaties voor grondstoffen groter dan die voor afgewerkte producten? We zien het antwoord in figuur~\ref{fig:vergelijkingTweeMarkten}. Links markt voor grondstoffen, rechts afgewerkte producten, met elk dezelfde 3 vraagcurves. $V_0$ stelt een normale, neutrale vraagcurve voor, $V_1$ die in een ``boomperiode'', wanneer het goed gaat en $V_2$ stelt de vraag voor in een recessieperiode. Het verschil tussen de twee markten is dus de aanbodcurve. $A_{grondstoffen}$ is minder prijselastisch.

Is dit verschil in elasticiteit logisch om te veronderstellen? Ja, want voor grondstoffen is het moeilijk om snel extra te produceren wanneer $p$ stijgt (extra gangen graven, nieuwe mijnen in gebruik nemen, etc.)

We zien dus dat het interval van mogelijke prijzen bij de markt voor grondstoffen groter is, maar dat het interval van mogelijke verhandelde hoeveelheden minder groot is dan bij de markt voor afgewerkte producten.



\subsection{Effecten van Overheidsoptreden}
Hoe kan de overheid de markt be\"invloeden wanneer ze de prijs te hoog/laag vindt en deze wil laten dalen/stijgen. Of wanneer ze vindt dat er te veel van een goed verhandeld wordt. 2 categorie\"en:
\begin{itemize}
	\item Niet-marktconforme maatregelen (verhinderen v/h totstandkomen v/e marktevenwicht)
    \item Marktconforme maatregelen (fin. prikkels geven om zo marktevenwicht te be\"invloeden)
\end{itemize}

\subsubsection{Maximumprijs}
Het is evident dat een maximumprijs hoger dan de marktprijs weinig zin heeft. Indien de maximumprijs onder de marktprijs gaat, ontstaat er een \textbf{vraagoverschot}. De prijs wordt dan dus wel gedrukt maar vele mensen vallen buiten de boot en het mist dus zijn sociale doel. Meestal onstaat er ook een \textbf{zwarte markt} samen met dit vraagoverschot. Er is namelijk een grote discrepantie tussen wat consumenten bereid zijn te betalen voor een extra eenheid en de prijs die prod. voor een extra eenheid vragen. (Figuur~\ref{fig:maximumprijs})
\begin{figure}[htbp]
	\centering
	\includegraphics[scale=0.4]{Images/white.png}
	\caption{Het effect van een maximumprijs}
	\label{fig:maximumprijs}
\end{figure}


\subsubsection{Minimumprijs}
Analoog aan maximumprijs, zal wederom leiden tot een zwarte markt.


\subsubsection{Quota's}
Aanbodcurve wordt verticaal vanaf $q_{max}$ waardoor $p$ zal stijgen. Dit zorgt er voor dat illegale activiteiten alleen nog maar extra incentief krijgen. (Figuur~\ref{fig:quota})
\begin{figure}[htbp]
	\centering
	\includegraphics[scale=0.4]{Images/white.png}
	\caption{Het effect van een quota}
	\label{fig:quota}
\end{figure}


\subsubsection{Indirecte Belastingen en Subsidies}
Bij belastingen of subsidies kunnen we niet meer spreken over ``de prijs'', we maken een onderscheid tussen volgende drie prijzen:
\begin{description}
	\item[marktprijs ($p^*$)] wat de ene partij aan de andere betaalt
    \item[consumentenprijs ($p^V$)] effectief betaalde prijs door de cons., rekening houdend met door hem betaalde belasting of ontvangen subsidies.
    \item[producentenprijs ($p^A$)] wat de producent effectief krijgt.
\end{description}

We maken ook nog onderscheid tussen \textbf{accijnzen}, een belasting van een vast bedrag per hoeveelheid, en \textbf{waardebelastingen}, die een percentage van het bedrag zijn (zoals BTW).

\begin{figure}[htbp]
	\centering
	\includegraphics[scale=0.4]{Images/white.png}
	\caption{Accijns vs. waardebelasting}
	\label{fig:belastingen}
\end{figure}
Merk op dat figuur~\ref{fig:belastingen} niet alle mogelijkheden weergeeft. Er zijn belastingen mogelijk voor zowel $A$ als $V$, alsook subsidies voor beiden. Denk altijd goed na in welke richting de curve zal bewegen bij het heffen van een belasting of subsidie.

We willen natuurlijk gaan kijken naar de uiteindelijke effecten van een belasting of een subsidie op een bepaalde markt.
\begin{figure}[htbp]
	\centering
	\includegraphics[scale=0.4]{Images/white.png}
	\caption{Het effect van een belasting}
	\label{fig:effectVanBelasting}
\end{figure}
$\rightarrow$ belasting zorgt ervoor dat beide partijen er slechter aan toe zijn, zoals je kan zien in figuur~\ref{fig:effectVanBelasting}. De overheid gaat er op vooruit. Dat de prod. er slechter aan toe is, is logisch. Een deel van de belasting wordt echter overgedragen aan de consumenten via het \textbf{afwentelingsmechanisme}. De reden dat dit gebeurt is prijselasticiteit van zowel de vraag als het aanbod.

Hoe het verband is tussen $\varepsilon^{V}_p$ en de afwenteling van een belasting zie je in figuur~\ref{fig:vergelijkAfwentelingVraag}:
\begin{figure}[htbp]
	\centering
	\includegraphics[scale=0.4]{Images/white.png}
	\caption{Vergelijken afwentelingsmechanisme bij perfect elastiche/inelastiche vraag}
	\label{fig:vergelijkAfwentelingVraag}
\end{figure}
We zien dus dat als de $V$ perfect inelastisch is, de afwenteling op de consumenten 100\% bedraagt aangezien $p^*$ stijgt met de waarde van de belasting ($t$). Bij een perfect elastische vraag komt de ganse belasting op de kap van de producent terecht, er is geen prijsstijging, maar zal $q$ afnemen.

In figuur~\ref{fig:vergelijkAfwentelingAanbod} zie je het afwentelingsmechanisme van een belasting op perfect (in)elastische $A$:
\begin{figure}[htbp]
	\centering
	\includegraphics[scale=0.4]{Images/white.png}
	\caption{Vergelijken afwentelingsmechanisme bij perfect elastich/inelastich aanbod}
	\label{fig:vergelijkAfwentelingAanbod}
\end{figure}
Hier zien we dat hoe elastische $A$ is, hoe kleiner het deel is dat de producenten van de belasting zullen moeten betalen. Vergeet in de tekening met perfect inelastische $A$ zeker niet de producentenprijs aan te duiden zodat je kan zien dat zij de hele belasting dragen.

Het maakt ook niet uit op wie een belasting geheven wordt. Het uiteindelijke resultaat van een belasting op $A$ of op $V$ is identiek, gegeven dat $t$ in beide gevallen ook hetzelfde was. Hoe de grafieken er uit zien na de belasting is niet hetzelfde, maar eens je $p^A$ en $p^V$ aanduidt, zou je moeten zien dat deze voor beide gevallen identiek zijn. Het resultaat staat in figuur~\ref{fig:vergelijkBelasting}.

\begin{figure}[htbp]
	\centering
	\includegraphics[scale=0.4]{Images/white.png}
	\caption{Vergelijken van situatie na belasting op V en op A}
	\label{fig:vergelijkBelasting}
\end{figure}
