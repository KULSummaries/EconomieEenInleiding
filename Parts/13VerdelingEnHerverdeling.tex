\section{Verdeling en Herverdeling}
\subsection{Beschikbaar Inkomen en Welvaart}
Pareto-effici\"entie geeft hoogst mogelijke welvaart, maar welvaart kan zeer scheef verdeeld zijn. Ongelijkheid zullen we meten door naar het inkomen te kijken. Meer bepaald het \textbf{beschikbaar gezinsinkomen}. Dit is het \textbf{vrij besteedbaar} inkomen v/h gezin, en is dus dat deel, dat aan consumptie of sparen gespendeerd kan worden. Er zijn geen andere plichten meer die er op wegen.

We komen aan dit beschikbaar gezinsinkomen door te vertrekken van alle \textbf{primaire inkomensbronnen} en van dat belastbaar inkomen de belastingen af te trekken (zie slide 3). We bekomen dan het beschikbaar inkomen.

Slide 4 kijkt naar de 5 kwintielen van de bevolking, en wat hun voornaamste bronnen van inkomen zijn. We zien daarin dat de sociale zekerheid herverdelend werkt.

Is het beschikbaar gezinsinkomen wel een goede maatstaf? Gezinsgrootte speelt ook en rol. We moeten dus kijken naar \textbf{inkomen per capita}, of \textbf{equivalentieschalen}, deze gaan nog wat verder (zie slide 6).

\subsection{Ongelijkheid}
Een eerste methode om naar inkomensongelijkheid te kijken is het opstellen van verschillende inkomenscategorie\"en en te kijken hoe de bevolking hierover verdeeld is (zie slide 8 \& 9). Voor Belgi\"e zien we dat dit een rechts scheve verdeling oplevert.

Een andere manier is het bekijken van de decielen en welk deel van de totale inkomsten dat bevolkingsdeel heeft. Dit kan uitgezet worden op een \textbf{Lorenz-curve}, waarbij de diagonaal de perfecte verdeling voorstelt. De oppervlakte tussen deze diagonaal en de Lorenz-curve zelf is dan de ongelijkheid. Deze oppervlakte kan uitgedrukt worden met de \textbf{Gini-co\"effici\"ent}. Deze is gelijk aan:
\begin{equation}
   \frac{L}{1/2} = \frac{1/2 - M}{1/2} = 1-2M
\end{equation}
waarbij $L$ de oppervlakte tussen de diagonaal en de Lorenz-curve is, en $M$ de oppervlakte onder de Lorenz-curve. Dit getal is 0 bij perfecte gelijkheid en 1 bij perfecte ongelijkheid. Met behulp van dit getal kunnen we ongelijkheden tussen landen of over de tijd gaan vergelijken.

Soms zegt een Lorenz-curve of Gini-co\"effici\"ent ook niet alles. Neem als voorbeeld figuur~\ref{fig:tweeLorenzCurves}. Op zich is A gelijker verdeeld, maar bij B hebben de laagste inkomens een groter aandeel van het geheel. Het is moeilijk te zeggen welke de ``beste'' van de twee is.
\begin{figure}[htbp]
   \centering
   \includegraphics[scale=0.4]{Images/white.png}
   \caption{Twee Lorenz curves (slide 13)}
   \label{fig:tweeLorenzCurves}
\end{figure}

Er zijn ook nog alternatieven voor de Gini-co\"effici\"ent, de \textbf{percentielratio's}.

Er zijn een aantal mogelijke oorzaken voor ongelijkheid:
\begin{itemize}
   \item Minder herverdeling
   \item Sociodemografische veranderingen: bijvoorbeeld meer alleenstaanden
   \item Verdeling van primair inkomen wordt ongelijker. Decielratio's zijn handig om dit te bekijken (slide 22).
\end{itemize}

De Gini-co\"effici\"ent en Lorenz-curve zeggen echter niet veel over armoede. Stel dat iedereen in een land plots het dubbele zou verdienen, blijft de verdeling toch volledig hetzelfde.

\subsection{Armoede}
Kan gedefini\"eerd worden op twee manieren:
\begin{description}
   \item[Absolute armoedegrens]: vb de grens van \'e\'en dollar per dag
   \item[In relatieve zin]: inkomen laat niet toe volwaardig aan normaal maatschappelijk leven mee te doen
\end{description}
Er zijn verschillende methodes om zo'n relatieve armoedegrens te bepalen (slide 23). Het probleem daarmee is dat ze allemaal arbitrair zijn, maar het blijft wel relatief waardoor er over landen heen vergeleken kan worden.

Merk ook op dat de Gini-co\"effici\"ent misschien niets over armoede, maar als je landen gaat rangschikken volgens deze Gini-co\"effici\"ent en de armoede zie je wel dat er een zekere correlatie bestaat.
