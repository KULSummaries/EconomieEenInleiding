\section{Onvolmaakte Mededinging en Productdifferentiatie}

\subsection{Homogeen Oligopolie}
Een \textbf{oligopolie} is een marktvorm waarbij er wederzijdse be\"invloeding is $\Rightarrow$ gedrag mede bepaald door gedrag v/d concurrentie. Een \textbf{homogeen oligopolie} is een oligopolie waarin het product homogeen is.

We beginnen met het bespreken van een \textbf{homogeen duopolie}, waarbij beide ondernemingen dezelfde kostenstructuur hebben. Op het eerste zicht lijkt het een goed idee voor beiden om samen te werken en een kartel te vormen waarbij ze een $q^*$ bepalen die er voor zorgt dat hun winst zo hoog mogelijk is.
\todo[inline]{TODO: berekening slide 4 invoeren}
%\begin{align*}
%
%\end{align*}

We komen een evenwicht uit, maar beide spelers zullen een incentief hebben om hier van af te wijken. We maken eerst de berekening als A vals speelt en B zich aan de afspraak zal houden, maar ze zullen beiden meteen afwijken van de afspraak. Deze situatie berekenen we ook.
\todo[inline]{TODO: berekeningen slide 6 en 8 invoeren}
%\begin{align*}
%
%\end{align*}

Het uiteindelijke evenwicht dat zal onstaan, indien beide spelers \textbf{simultaan beslissen}, is het \textbf{Cournot-evenwicht}. Om dit evenwicht te bekomen zullen we de \textbf{reactiefunctie} voor beide spelers opstellen. Deze lossen we dan op in een stelsel en we bekomen het evenwicht.

Meestal zullen ondernemingen niet simultaan beslissen, maar zijn er een leider en een of meerdere volgers op de markt, waarbij de leider als eerste zijn hoeveelheid zal gaan zetten. De leider kent wel de reactiefuncties van de volgers en zal hier dus op anticiperen. Op basis van deze reactiefunctie(s) kan de leider dan een functie opstellen waarin enkel zijn eigen productie variabel is. Hij kan deze functie dan gaan maximaliseren. Het evenwicht dat hieruit ontstaat noemt men het \textbf{Stackelberg-evenwicht}.


\subsection{Oligopolie en Speltheorie}

\subsection{Productdifferentiatie}
\subsubsection{Monopolistische Mededinging}
=geen homogene goederen(productdifferentiatie) = markt van gelijkaardige goederen die hetzelfde nut vervullen.

\subsubsection{Heterogeen Oligopolie: Model van Hotelling}
