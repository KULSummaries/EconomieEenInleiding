\section{Macro-Economische Analyse: Wat en Waarom?}
Iets waar in de micro-economie geen rekening mee gehouden wordt, is de rol van geld en de inflatie die dit met zich meebrengt. Een ander verschil is dat we niet zullen kijken naar de vraag van een individuele consument, maar naar de geaggregeerde vraag (en aanbod). Deze andere benadering om naar de economie te kijken leidt tot andere inzichten dan die uit de micro-economie voortkomen. Drie voorbeelden:
\begin{itemize}
    \item \textbf{De wet van Say}: Als we teruggrijpen naar de economische kringloop (figuur~\ref{fig:economischeKringloop}), zien we dat deze ``gesloten'' is. Al het geld dat de bedrijven als loon aan de arbeiders betalen komt uiteindelijk ook terug bij hen terecht.
    \begin{figure}[htbp]
        \centering
        \includegraphics[scale=0.4]{Images/white.png}
        \caption{De economische kringloop (14.2)}
        \label{fig:economischeKringloop}
    \end{figure}

    Stel nu dat alle prijzen stijgen. Volgens de micro-economie (wet van de Vraag) zal $q$ hierdoor afnemen. In de macro-economie is dit niet noodzakelijk het geval. De gezinnen krijgen nu ook een hoger inkomen, doordat bedrijven meer geld binnen krijgen, dat op zijn beurt weer terug naar de ondernemingen vloeit. We weten dus niet wat $V$ zal doen. Bijgevolg kan $q$ toe- of afnemen.

    $\Rightarrow$ Wet van Say: \textit{Elk aanbod schept haar eigen vraag.}

    \item \textbf{Het gebruik van geld}: wederom grijpen we terug naar de economische kringloop (figuur~\ref{fig:economischeKringloop}). Beide pijlen in deze kringloop geven dezelfde onderliggende economische realiteiten weer en hebben dus dezelfde waarde.
    $\Rightarrow$ Is geld neutraal? Dit leidt tot de \textit{Fischervergelijking}:
    \begin{equation}
        M \times V = P \times Q
    \end{equation}
    Waarbij $M$ de geldhoeveelheid is, $V$ de omloopsnelheid van het geld, $P$ het algemeen prijsniveau en $Q$ de geaggregeerde geproduceerde hoeveelheid in de economie. Deze vergelijking stelt dus dat een wijziging in $M$ of $V$ dus een gevolg moet hebben in de re\"ele economie.

    Andere manieren waarop geld de re\"ele economie kan verstoren is via \textit{inflatie} en het \textit{spaarlek}.

    \item \textbf{Informatie- en Co\"ordinatieproblemen}: In de macro-economie gaat het prijsmechanisme als co\"ordinatiemechanisme vaak niet goed werken door \textit{informatieproblemen} omdat de realiteit vaak sterk afwijkt van het model van perfecte mededinging. Twee voorbeelden:
    \begin{itemize}
        \item \textbf{Investeringen}: De markt wordt gekenmerkt door optimisme en pessimisme (\textit{animal spirits}). Als een marktleider veel gaat investeren zullen vele kleinere spelers ook volgen omdat ze er van uitgaan dat de marktleider meer informatie over de markt heeft. De beslissing van \'e\'en speler om te investeren kan dus een kettingreactie veroorzaken. Het omgekeerde kan ook waar zijn.

        Deze animal spirits ontstaan dus door het gebrek aan informatie bij economische agenten. Dit inzicht cre\"eerde de tegenstelling tussen de klassieke economen (minimale overheidsinmenging) en de Keynesianen (overheid moet markt sturen en consumenten vertrouwen geven als het slecht gaat).

        \item \textbf{Spaarparadox}: sparen = niet of uitgesteld consumeren en spaargedrag is afhankelijk van de verwachtingen. Pessimisme op de markt zal sparen doen toenemen. Dit zal echter leiden tot het afnemen van de inkomsten van de bedrijven, wat ook zorgt dat minder geld terug naar de consumenten vloeit waardoor in het algemeen het inkomen afneemt. Het pessimisme wordt dus bevestigd. Dit noemt men de \textit{spaarparadox van Keynes}.
    \end{itemize}
\end{itemize}

\subsection{Keynes Vs. Klassiekers}
\begin{description}
    \item[Klassieke economen]:
    \begin{itemize}
        \item Prijsveranderingen zorgen dat verschillen tussen $A$ en $V$ automatisch weggewerkt worden.
        \item Wet van say zegt dat er geen landurige onevenwichten op macro-economisch niveau kunnen zijn.
    \end{itemize}

    \item[Keynes]:
    \begin{itemize}
        \item Animal spirits kunnen er voor zorgen dat het evenwicht zich niet altijd vanzelf herstelt.
        \item Permanente werkloosheid kan in het evenwicht aanwezig zijn.
        \item Prijzen veranderen gestaag en zullen hun co\"ordinerende rol maar moeilijk kunnen vervullen.
    \end{itemize}
\end{description}
