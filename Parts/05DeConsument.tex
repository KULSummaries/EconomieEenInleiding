\section{De Consument}

Consumenten nemen beslissingen. Vb. aanbieden van arbeid of niet, wat te kopen en hoeveel van wat(budgetbesteding).

\subsection{De Budgetbeperking}
We gaan voor de gemakkelijkheid werken met een budgetrechte waarin 2 goederen voorkomen. Deze is namelijk grafisch makkelijke voor te stellen, maar is makkelijk uitbreidbaar naar meerdere goederen.
\begin{equation}
	p_1 q_1 + p_2 q_2 \le y
\end{equation}

We kunnen deze budgetbeperking in een grafiek (figuur~\ref{fig:budgetBeperking}) weergeven. Deze figuur doet veel denken aan sectie~\ref{ssec:SpecialisatieEnRuil} over specialisatie en ruil, waar we ook de relatieve kost op een grafiek konden tekenen via zo'n rechte. Figuur~\ref{fig:budgetBeperking} kan dus op een gelijkaardige manier ge\"interpreteerd worden.

\begin{figure}[htbp]
	\centering
	\includegraphics[scale=0.4]{Images/white.png}
	\caption{De budgetbeperking}
	\label{fig:budgetBeperking}
\end{figure}

Het spreekt voor zich dat deze ganse rechte zal verschuiven wanneer het inkomen ($y$) toe- of afneemt. De curve zal kantelen wanneer de prijs van \'e\'en van de goederen af- of toeneemt. Ze zal ook kantelen als beide prijzen veranderen, maar niet in dezelfde mate. Indien de prijzen in dezelfde mate veranderen zal dit hetzelfde effect als het toe- of afnemen van het inkomen hebben.

\subsection{Voorkeuren}
Bij voorkeuren van consumenten gaan vertrekken we van het vergelijken van goederenbundels, door die consument. Dit heeft drie eigenschappen:
\begin{itemize}
	\item elke goederenbundel moet geordend moeten worden (beter dan, slechter dan of equivalent aan een andere)
    \item deze ordening is transistief. Als A beter is dan B en B beter dan C, is A ook beter dan C enzovoorts.
    \item de consument is niet-verzadigd, meer is altijd beter.
\end{itemize}

$\rightarrow$ indifferentie-curves met erboven alle bundels die beter zijn en er onder alle bundels die slechter zijn voor een bep. consument. Alle bundels op eenzelfde indifferentiecurve zijn equivalent.

Hoe hoger de indifferentiecurve waarop een bundel ligt, hoe hoger het nut van die bundel voor die persoon. Indifferentiecurves hangen van persoon tot persoon af.

Uit transitiviteit en niet-verzadiging volgt dat indifferentiecurves nooit kunnen snijden.

Een indifferentiecurve zal dalend zijn aangezien bundels equivalent zijn wanneer een bij minder van het ene, meer van het andere in de plaats verwacht wordt. Daarenboven worden indifferentiecurves meestal convex afgebeeld omwille van de assumpties die genomen worden m.b.t. de $MSV$.

\subsubsection{De Marginale Substitutievoet (MSV)}
=de prijs van het ene goed uitgedrukt in het andere goed. M.a.w. als je van het ene goed moet inleveren, hoeveel wil je van het andere als compensatie om hetzelfde nut te behouden. Bijgevolg geldt dat:
\begin{equation}
	MSV = \frac{\Delta q_1}{\Delta q_2}
\end{equation}
bij een constant nut. In een bepaald punt geldt dan uiteraard weer dat de MSV de limiet hiervan is, en bijgevolg dus de afgeleide in dat punt op de indifferentiecurve.

We nemen dus aan dat de absolute waarde van de $MSV$ stijgt wanneer we naar rechts beneden bewegen op de indifferentiecurve, aangezien we dan reeds veel van $q_1$ hebben en dus weinig een extra eenheid $q_1$ steeds minder en minder waard wordt.

\subsection{Keuze van de Consument}
=combinatie budgetrechte en ``indifferentie-kaart''. We zoeken het punt op de budgetrechte dat op een zo hoog mogelijke indifferentiecurve ligt, of in andere woorden, waar de budgetrechte de raaklijn aan een indifferentiecurve is. In dat punt zal ook de $MSV$ gelijk zijn aan de relatieve prijs:
\begin{equation}
	MSV = \frac{-p_1}{p_2}
\end{equation}

Een verbeterbare keuze kan namelijk geen evenwicht zijn. Waneer $|MSV| > \frac{p_1}{p_2}$ ben je bereid meer te betalen van goed 2 voor een extra eenheid van goed 1 dan dat je eigenlijk moet betalen $\rightarrow$ hoger nut bereikbaar want meer is beter. Dit kan je ook zien in figuur~\ref{fig:verbeterbareKeuze}.
\begin{figure}[htbp]
	\centering
	\includegraphics[scale=0.4]{Images/white.png}
	\caption{Een keuze die verbeterd kan worden}
	\label{fig:verbeterbareKeuze}
\end{figure}

\subsection{Verschuivingen van het Evenwicht}

Het evenwicht zal veranderen wanneer het inkomen van een individu wijzigt, of wanneer de relatieve prijs van de goederen veranderd (of wanneer de voorkeuren wijzigen). Het nieuwe evenwicht zal dan hoogstwaarschijnlijk op een nieuwe indifferentiecurve liggen. Uit de verandering van $q_1$ en $q_2$ kan je ook afleiden of het over een normaal goed of inferieur goed gaat (zie slide 16 en 17).

Bij een verandering van de relatieve prijs spreken we ook over een \textbf{substitutie-effect} en \textbf{inkomens-effect}. Deze twee samen maken de totale verandering van $q_1$ of $q_2$ uit. Het substitutie effect is altijd positief voor het goed dat goedkoper geworden is. Door het feit dat het minder kost, zal er meer van gekocht worden, denk hierbij aan meer is beter. Het inkomenseffect wil zeggen dat je op een nieuwe indifferentiecurve terecht zal komen. Het is namelijk zo dat als je na de prijswijziging een bundel wilt kopen die je voorheen kon kopen, je nu geld over zal hebben en je dus een hoger nut kan bereiken, ofwel dat je te weinig geld hebt en dat je het met minder zal moeten stellen. Het inkomens- en substitutie-effect zijn grafisch weergeven in figuur~\ref{fig:inkomensEnSubstitutieEffect}.
\begin{figure}[htbp]
	\centering
	\includegraphics[scale=0.4]{Images/white.png}
	\caption{Inkomens- en substitutie-effect grafisch weergegeven}
	\label{fig:inkomensEnSubstitutieEffect}
\end{figure}

Tot slot zijn er nog twee speciale soorten goederen. \textbf{snobgoederen} zijn goederen waarvan er meer gevraagd wordt naarmate de prijs stijgt. Bijgevolg is $\varepsilon_{p}^V > 0$. 
\textbf{Giffengoederen} zijn goederen waarvoor de prijs daalt, maar er ook minder van geconsumeerd wordt, het zijn dus inferieure goederen en hebben dus een negatief inkomenseffect wanneer hun prijs daalt. Daarenboven is het bij Giffengoederen zo dat dit negatieve inkomenseffect het substitutie-effect volledig teniet doet. In tegenstelling inferieure goederen die geen giffengoed zijn, zal de vraagcurve voor giffengoederen stijgend verlopen. Bij giffengoederen is het zo dat $\varepsilon_{p}^V > 0$ en dat $\varepsilon_{y}^V < 0$. Het is dus een ``inferieur snobgoed''. Beschouw volgend overzicht (let goed op waar het $\varepsilon_{y}^V$ of $\varepsilon_{p}^V$ is):
\begin{description}
	\item[Inferieur goed]: $\varepsilon_{y}^V < 0$
    \item[Normaal goed]: $\varepsilon_{y}^V > 0$
    \item[Noodzakelijk goed]: $0 < \varepsilon_{y}^V < 1$
    \item[Luxegoed]: $\varepsilon_{y}^V > 1$
    \item[Snobgoed]: $\varepsilon_{p}^V > 0$
    \item[Giffengoed]: $\varepsilon_{p}^V > 0$ en $\varepsilon_{y}^V < 0$
\end{description}


\subsection{Van Individuele Vraag naar Marktvraag}

Wanneer we de prijs van een goed gradueel veranderen in het model dat we hierboven besproken en de evenwichten daarin aanduiden bekomen we de vraagcurve voor een bepaalde consument. Deze kunnen we dan aggregeren tot een marktvraag. We nemen daarvoor een bepaalde $p$ en kijken hoeveel er voor die $p$ gevraagd wordt. Zo kunnen we de ganse curve opbouwen.



