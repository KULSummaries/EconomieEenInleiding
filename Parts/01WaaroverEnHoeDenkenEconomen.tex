\section{Waarover en hoe Denken Economen?}
Twee vragen centraal in dit hoofdstuk:
\begin{itemize}
	\item Hoe denken economen?
    \item Waarover denken economen?
\end{itemize}

Waarover gaat economie? Een aantal definities:
\begin{description}
	\item[Robbins (1932)] \textit{Gedrag van mensen bestuderen. Gedrag om bepaald doel te bereiken met beperkte middelen.} Beperkte middelen $\rightarrow$ kiezen.
    \item[Marschall (1890)] \textit{Streven naar welvaart/welzijn.} $\rightarrow$ economie = het onderzoeken van de acties die uit dit streven leiden.
\end{description}

\subsection{Productie als Voorbeeld van hoe Economen Denken}
De laaste 100 jaar is de totale wereldproductie (BBP) exponenti\"{e}el gestegen. We willen dit gaan meten en vergelijken maar stuiten op twee problemen:
\begin{itemize}
	\item we moeten alles op een gemeenschappelijke munt zetten.
    \item \textit{intertemporeel}: waarde van totale prod. kan wijzigen door zowel volume- als prijswijzigingen. We zijn echter alleen ge\"{i}nteresseerd in de volumewijzigingen.
\end{itemize}
$\rightarrow$ we vergelijken alles met het BBP in USD van een bepaald basisjaar. (We werken met re\"{e}ele waarden.)

De economie is dus enorm vooruit gegaan (zie slides 8-15 voor grafieken), er zijn drie grote conclusies, de productiekorf:
\begin{itemize}
	\item is veel groter geworden
    \item is anders samengesteld
    \item wordt op een andere manier vervaardigd
\end{itemize}

Cruciale vragen hierbij zijn hoeveel, wat, hoe, voor wie en door wie er geproduceerd wordt. Antwoord zal grotendeels in markt-/prijsmechanisme liggen, maar soms ook rol voor overheid.

\subsection{Hoe Kijken Economen naar deze Vragen?}
(Zie slide 17) De belangrijkste invalshoek is die van \textbf{schaarste}.  Er zijn vele behoeften, quasi oneindig en de meeste zijn herhaalbaar, er zijn er ook die vermenigvuldigbaar\todo{wat is herhaalbaar en vermenigvuldigbaar?} zijn\\
$\rightarrow$ om behoeften te te vervullen zijn goederen nodig. Beschikbare goederen zijn niet voldoende om aan alle behoeften de voldoen \textbf{= schaarste}.

Dit leidt tot kiezen en ook tot de problemen van wat en hoe te produceren.

Uit deze schaarste, en de keuze die daardoor gemaakt moeten worden leiden een aantal dingen zoals effici\"{e}ntie, opportuniteitskost, specialisatie + arbeidsverdeling, ruil + markt en rationeel handelen + evenwicht.


\subsection{Centrale Begrippen van de Economische Analyse}
We kijken dus naar de productie. Er worden een aantal concepten gelanceerd om hier verder in te kunnen gaan. De eerste drie zijn \textbf{economische agenten}. Deze zijn personen en instellingen die beslissingen nemen (binnen een econ. context).

\subsubsection{Onderneming}
(zie slide 18) Een onderneming gaat output produceren a.d.h.v. inputs. Input die volledig in de output ``verdwijnt'' noemt men \textbf{lopende inputs}. Dit kan gaan over grond- of hulpstoffen. Anderzijds zijn er \textbf{meer duurzame inputs} zoals kapitaal (machines, etc.) en arbeid(menselijk kapitaal). Deze worden ook in het productieproces ingezet. De onderneming gaat \textbf{toegevoegde waarde} ($TW$) te cre\"{e}ren in het productieproces aangezien de som van de inputs beter geschikt is voor behoeftebevrediging dan de afzonderlijke inputs alleen.

\subsubsection{Consumenten}
Waarschijnlijk de belangrijskte econ. agenten. Ze bieden productiefactoren aan, via arbeid (\textit{L}) en krijgen hiervoor een inkomen (\textit{y}). Dit kunnen ze dan besteden aan consumptie of aan sparen (= toename v/h vermogen = uitgesteld sparen).

\subsubsection{Overheid}
De overheid heeft 4 grote taken:
\begin{description}
	\item[Regulering] $\rightarrow$ econ. relaties vlotter laten verlopen (vb. via duidelijk eigendomsrecht).
    \item[Inkomensverdeling] zodat mensen die niet deel kunnen nemen aand prod. proces ook een inkomen hebben.
    \item[Prod. v. publieke goederen] pub. goederen = goederen waarvan niemand uitgesloten kan worden van consumptie (vb. straatverlichting), of consumptie is niet-rivaal (vb. straatverlichting)
    \item[Sturen econ. activiteit] soms is bepaalde prod. niet optimaal $\rightarrow$ ingrijpen
\end{description}

\subsubsection{Een Eenvoudige Economische Kringloop}
Gezinnen leveren arbeid aan ondernemingen (fysieke stroom) en krijgen hiervoor een inkomen (geldstroom). Anderzijds leveren de ondernemingen goederen en diensten aan de gezinnen (fysieke stroom) en krijgen zij hier voor betaald (geldstroom), als weergeven in figuur~\ref{fig:eenvoudigeEconomischeKringloop}.

Deze is echter niet erg realistisch:
\begin{itemize}
	\item Er bestaan nog econ. agenten
    \item Geen import/export
    \item gezinnen sparen niet $\rightarrow$ geen ``lek'' in de kringloop
\end{itemize}

\begin{figure}[htbp]
	\centering
	\includegraphics[scale=0.4]{Images/white.png}
	\caption{Een eenvoudige economische kringloop}
	\label{fig:eenvoudigeEconomischeKringloop}
\end{figure}


\subsection{Productiviteit}
Niet alleen de wereldproductie is enorm toegenomen, maar ook de productiviteit is enorm toegenomen en lag aan de basis van die stijging in wereldproductie. Met die stijging in productiviteit bedoelen we dat met dezelfde inzet van middelen nu meer geproduceerd kan worden dan voorheen.

Aangezien productiviteitsstijgingen de basis vormen van welvaartstoenamen, is dit een belangrijk thema binnen de economie. Redenen voor een stijging kan bvb. technologische vooruitgang zijn. Welvaartstoename kan ook via andere wegen tot stand komen, bvb via ruil na specialisatie.

\subsection{Specialisatie \& Ruil}
\label{ssec:SpecialisatieEnRuil}

Welvaart kan toenemen via specialisatie en ruil. Dit doordat specialisatie leidt tot een verhoogde arbeidsproductiviteit. Verder doorgedreven specialisatie leidt ook tot toenemende ruilactiviteiten.

Veronderstel twee partijen (Bart en Lisa), die elk twee goederen produceren (ipods en hemden). Overige assumpties:
\begin{itemize}
	\item arbeid enige productiefactor
    \item gemiddelde arbeidsprod. constant
    \item beschikbare arbeidsuren gelijk voor beide partijen (4u/dag in dit vb.)
\end{itemize}

Veronderstel volgende tijden dat Bart en Lisa doen over het produceren van 1 hemd en 1 ipod:

\begin{table}[h]
	\centering
    \begin{tabular}{ | c |  c |  c | }
		\hline
		& Hemd  &  Ipod \\
		\hline
		 Lisa & 20 min & 10 min \\
		\hline
		Bart & 10 min & 20 min\\
		\hline
	\end{tabular}
	\caption{Tijd voor Bart en Lisa om een Ipod of Hemd te maken}

\end{table}

Als we dit omzetten naar productie per uur komen we volgende tabel uit:

\begin{table}[h]
	\centering
    \begin{tabular}{ | c |  c |  c | }
		\hline
		& Hemd  &  Ipod \\
		\hline
		 Lisa & 3/uur & 6/uur \\
		\hline
		Bart & 6/uur & 3/uur\\
		\hline
	\end{tabular}
	\caption{Output per uur voor Lisa en Bart}
\end{table}

We kunnen nu al op zicht zien wie beter is in de productie van welk goed en daar dus ook in gaat specialiseren, maar we kunnen het ook nog op een algemenere manier zien voor wanneer het niet zo duidelijk is:

\begin{table}[h]
	\centering
    \begin{tabular}{ | c |  c |  c | }
		\hline
		& Hemd  &  Ipod \\
		\hline
		 Lisa & 2 Ipod & 1/2 Hemd \\
		\hline
		Bart & 1/2 Ipod & 2 Hemden\\
		\hline
	\end{tabular}
	\caption{Comparatieve kosten van hemden en Ipods voor Lisa en Bart}
    \label{table:sec1:compCost}
\end{table}

Als je naar tabel~\ref{table:sec1:compCost} kijkt zie je dat Lisa en Bart elk gaan produceren wat voor hen relatief gezien het goedkoopst is. In andere woorden, ze gaan produceren waar ze het best in zijn. De ruilvoet voor Ipods of Hemden die bij handel tot stand komt zal tussen de relatieve kost voor Ipods of Hemden van Bart en Lisa liggen.

We kunnen dit alles ook via grafieken weergeven. De Productiemogelijkhedencurve ($PMC$) in autarkie is in figuur~\ref{fig:cmcEnPmcAutarkie} weergegeven. Het feit dat de relatieve kost van beide goederen niet gelijk is voor Bart en Lisa kan je zien aan de verschillende helling van de curves. In dit geval, aangezien er geen handel is, is de $PMC = CMC$, waarbij dat laatste voor consumptiemogelijkhedencurve staat. M.a.w., hun eigen productie is wat ze kunnen consumeren.
\begin{figure}[htbp]
	\centering
	\includegraphics[scale=0.4]{Images/white.png}
	\caption{CMC en PMC voor handel van Bart en Lisa in autarkie}
	\label{fig:cmcEnPmcAutarkie}
\end{figure}

Wanneer er handel onstaat, komt daarbij ook een bepaalde ruilvoet. Wanneer we deze kennen, kunnen we de nieuwe $CMC$ tekenen, als weergeven in figuur~\ref{fig:cmcHandel} voor een ruilvoet van 1:1. Je kan nu ook grafisch zien dat ze erop vooruitgaan qua consumptie.
\begin{figure}[htbp]
	\centering
	\includegraphics[scale=0.4]{Images/white.png}
	\caption{CMC en PMC voor handel van Bart en Lisa na handel}
	\label{fig:cmcHandel}
\end{figure}

Tot slot kunnen we ook nog de $PMC$ van de hele samenleving tekenen (Bart en Lisa samen). Het resultaat daarvan zie je in figuur~\ref{fig:cmcHandel}.


\subsection{Hoe Spreken Economen?}
Lijkt onbelangrijk, lees in boek.
