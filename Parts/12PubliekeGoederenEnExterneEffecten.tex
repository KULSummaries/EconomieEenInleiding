\section{Publieke Goederen en Externe Effecten}

Soms faalt de markt als je ze haar vrije gang laat gaan, bvb in het geval van publieke goederen en externe effecten.

\subsection{Publieke Goederen}

Een zuiver publiek goed heeft twee eigenschappen:
\begin{description}
   \item[Niet Uitsluitbaar]: beschikbaar voor iedereen wanneer het geproduceerd wordt en niemand kan van consumptie worden uitgesloten.
   \item[Niet-Rivaliteit in Consumptie]: De marginale kost om \'e\'en extra individu aan te bieden is 0. Het kost bijvoorbeeld niets als er \'e\'en extra individu achter een dijk komt bijwonen. Ook zal als dit gebeurt niet meer geld aan straatverlichting gegeven moeten worden.
\end{description}

Goederen kunnen ook slechts \'e\'en van deze eigenschappen hebben, maar daar gaan we later op in.

De marktvraag voor een \textbf{privaat} goed = horizontale sommatie van de individuele gevraagde hoeveelheden. Bij de marktvraag voor een \textbf{publiek} goed ligt dat anders. Daar is dat gelijk aan de \textbf{verticale sommatie} van de individuele gevraagde hoeveelheden, want als iedereen 10 eenheden vraagt, zijn 10 eenheden voldoende.

Een vb kan je zien in figuur~\ref{fig:voorbeeldPubliekGoed}. Je kan zien dat de $MMB$ (\textit{Maatschappelijke Marginale Betalingsbereidheid}) de verticale som is van de $MB$ van Bart en Lisa.

\begin{figure}[htbp]
   \centering
   \includegraphics[scale=0.4]{Images/white.png}
   \caption{Voorbeeld verticale sommatie (slide 7)}
   \label{fig:voorbeeldPubliekGoed}
\end{figure}

Een effici\"ent, pareto-optimaal, niveau van voorziening moet voldoen aan de \textbf{Samuelsonregel}. Deze regel stelt dat:
\begin{equation}
   MB_L(q^*) + MB_B(q^*) = MK(q^*)
\end{equation}
en enkel in dit punt zal men een Pareto-effici\"ent punt bereiken. Hier wordt namelijk uitgegeven wat de hele maatschappij samen bereid is om af te staan.

Iedereen zal dus hetzelfde consumeren, maar de marginale bereidheid tot betalen is anders voor elk individu. Dit staat in tegenstelling tot private goederen, waar de geconsumeerde hoeveelheid verschilt per individu, maar iedereen wel op hetzelfde punt qua bereidheid tot betalen zit.

Stel dat we een pareto-effici\"ent punt vinden voor een publiek goed, zal dit dan ook effectief geproduceerd worden? Waarschijnlijk niet. Het \textbf{vrijbuitersprobleem} speelt hier een belangrijke rol in. Zie tabel~\ref{tab:vrijbuitersprobleem} voor hoe het zich manifesteert.

\begin{table}
   \centering
   \begin{tabular}[htbp]{cccc}
       & &  \multicolumn{2}{c}{Andere Bewoners} \\
       & & betalen & niet betalen \\
      \multirow{2}{*}{Landbouwer} & betalen & 3 & 1 \\
       & niet betalen & 4 & 2 \\
   \end{tabular}
   \caption{Voorbeeld vrijbuitersprobleem}
   \label{tab:vrijbuitersprobleem}
\end{table}
We zien dat er een dominante stratetie is voor elk individu om niet te betalen. Iedereen heeft er wel baat bij dat het publiek goed er komt, maar toch zal dit niet gerealiseerd worden als enkel de markt speelt. Bijgevolg zal \textbf{overheidsingrijpen} nodig zijn. Dit wil zeggen dat iedereen mee betaalt en de overheid zorgt dat het goed er komt. Iedereen moet bijdragen omdat de overheid geen onderscheid kan maken tussen de mensen die het goed echt niet willen en diegenen die doen alsof ze het niet willen (de vrijbuiters).

Zoals eerder reeds aangehaald kunnen sommige goederen ook slechts \'e\'en eigenschap van publieke goederen hebben. Er onstaan dan verschillende types goederen, weergeven in tabel~\ref{tab:overzichteigenschappenpg}.
\begin{table}[h]
   \centering
   \begin{tabular}[htbp]{|l|p{3cm} p{3cm} p{3cm}|}
      \hline
       & Uitsluitbaar & Gedeeltelijk Uitsluitbaar & Niet-Uitsluitbaar \\
       \hline
      Rivaal & private goederen & & Commons: vb visbestand \\
      Congestie & & quasipublieke goederen: vb zwembaden en snelwegen & \\
      Niet-Rivaal & Clubgoederen: vb internet &  & Zuiver publieke goederen \\
      \hline
   \end{tabular}
   \caption{Voorbeeld vrijbuitersprobleem}
   \label{tab:overzichteigenschappenpg}
\end{table}

Het type ``commons'' kent een zeer sterkt probleem van vrijbuiters. Niemand kan de toegang namelijk ontzeggen, maar alles wat je consumeert kan niet meer door anderen geconsumeerd worden.

Quasipublieke goederen krijgen ``problemen'' eens een bepaald congestieniveau is overschreden. Snelwegen op zich niet-rivaal, tenzij er te veel mensen zijn en dan is er file.

\subsection{Externe Effecten}
Externe effecten zijn de acties van een agent die positieve of negatieve effecten hebben op anderen, zonder dat deze agent daar voor beloond of gestraft wordt. De agent zal dus hier dus geen rekening mee houden aangezien hij er ook geen incentief voor heeft.

Voorbeelden hier van in consumptie: jezelf laten inenten heeft een positief effect op anderen aangezien je de ziekte niet meer gaat doorgeven. Een voorbeeld van een negatief extern effect is roken en het passief roken dat daar bij hoort.

Voorbeelden in productie: Een imker die zich ergens vestigt en waar de fruitboeren van profiteren. Negatief: mileuverontreiniging.

We kijken enkel naar negatieve externe effecten in productie.

Eerst wat termen:
\begin{description}
   \item{MMB} = maatschappelijke marginale baat
   \item{MB} = private marginale baat
   \item{MMK} = maatschappelijke marginale kost
   \item{MK} = private marginale kost
\end{description}

Dan geldt ook het volgende:
\begin{align*}
   MMB &= MB + \text{ externe baten}\\
   MMK &= MK + \text{ externe kosten}
\end{align*}

In figuur~\ref{fig:externEffect} zien we dat indien enkel de markt speelt, de producent geen rekening zal houden met de externe kosten en gewoon naar zijn eigen $MK$ zal kijken om zijn winst te maximaliseren. We zien ook dat er welvaartsverlies ontstaat doordat er meer geproduceerd wordt dan in het eigenlijk evenwicht, met de externe kosten mee in rekening gebracht, gevraagd wordt. Merk op dat er nog wel steeds een evenwicht kan zijn. Het is niet omdat er externe kosten zijn, dat de productie ook volledig wegvalt als die in rekening worden gebracht. Dit kan echter wel voorvallen als de $MMK$ curve volledig boven de vraagcurve ($MB$) ligt. Dan zal er geen evenwicht gevonden worden.

\begin{figure}[htbp]
   \centering
   \includegraphics[scale=0.4]{Images/white.png}
   \caption{Extern effect (slide 18 \& 19)}
   \label{fig:externEffect}
\end{figure}

Hoe ver zal een producent zijn productie dan terugdringen om vervuiling tegen te gaan? In figuur~\ref{fig:terugdringenExternEffect} stellen we dat een onderneming op een bepaald niveau van vervuilen zit, waarbij $MMK_l$, of de maatschappelijke marginale kost van het lozen overal gelijk is aan $MMB_{tv}$, de maatschappelijke marginale baat van het terugdringen van de vervuiling.

\begin{figure}[htbp]
   \centering
   \includegraphics[scale=0.4]{Images/white.png}
   \caption{Terugdringen extern effect (slide 22)}
   \label{fig:terugdringenExternEffect}
\end{figure}

Een onderneming heeft dan marginale kosten om deze vervuiling terug te dringen ($MK_{tv}$). Het optimale productieniveau bevindt zich dan op het punt waar $MK_{tv} = MMB_{tv}$.

De vraag is dan hoe je de markt naar dit ``juiste'' punt kan duwen. Er zijn een aantal mogelijkheden:
\begin{itemize}
   \item Wetgeving die de externe effecten internaliseert (enorm moeilijk)
   \item Uitstootnormen
   \item Milieuheffingen
   \item Verhandelbare emissierechten
\end{itemize}

\subsubsection{Uitstootnormen}
Men zou in theorie voor elke onderneming kunnen nagaan wat het punt is waar $MK_{tv} = MMB_{tv}$ en op basis daarvan een uitstoot norm instellen. Dit is echter moeilijk $\Rightarrow$ \textbf{uniforme uitstootnorm}. Dit brengt echter een aantal consequenties met zich mee, zoals we kunnen zien in figuur~\ref{fig:uniformeUitstootnorm}.

\begin{figure}[htbp]
   \centering
   \includegraphics[scale=0.4]{Images/white.png}
   \caption{Uniforme uitstootnorm (slide 26)}
   \label{fig:uniformeUitstootnorm}
\end{figure}

Stel dat de uitstootnorm ergens tussen de twee optimale productieniveau's ligt, dan zal er bij de ene onderneming nog meer welvaartswinst gemaakt kunnen worden, terwijl de andere al voorbij het optimum zit en reeds meer kosten maakt dan wat het de maatschappij oplevert. Zo'n quota is dus \textbf{welvaartsbevorderend}, maar niet helemaal optimaal, aangezien volgens de \textbf{Samuelsonregel} de $MK$'s aan elkaar gelijk zouden moeten zijn.

Een nodige voorwaarde voor het optimum is dus dat de marginale kosten, en dus de inspanning, voor elke onderneming gelijk zijn. Dit is het \textbf{equimarginaal kostenprincipe}. Als dit niet bereikt is zou voor dezelfde kostprijs een lagere uitstoot gerealiseerd worden. Stel dat een onderneming meer inspanning doet, tot een hogere $MK$ gaat m.a.w., dan zou een stuk van deze inspanning beter naar een andere onderneming gaan. Zij kunnen namelijk dezelfde reductie in vervuiling doorvoeren voor een lagere prijs. Dit is wederom een voorbeeld van \textbf{kosteneffici\"entie}.

\subsubsection{Belastingen}
Er zou een \textbf{outputbelasting} gehoffen kunnen worden die het externe effect reflecteert. Bijgevolg zal $MK$ dan samenvallen met $MMK$.

Indien de vervuiling goed te defini\"eren/meten valt, kan er ook een \textbf{belasting op de vervuildende factor} geheven worden.

Zoals in figuur~\ref{fig:outputBelasting} te zien is, wordt de belasting dan gelijk gesteld aan $MMB_{tv}$. De kost om de belasting terug te dringen is dan $MK$, maar het levert een grotere belastingsvermindering op, tot in het punt waar $MK = t$. De producent zal dus geneigd zijn de productie terug te schroeven tot in het optimum.

\begin{figure}[htbp]
   \centering
   \includegraphics[scale=0.4]{Images/white.png}
   \caption{Outputbelasting (slide 31)}
   \label{fig:outputBelasting}
\end{figure}

Dit punt voldoet dan aan het \textbf{equimarginaal kostenprincipe}, voor alle bedrijven zal gelden dat $MK = t$.

\subsubsection{Verhandelbare Emissierechten}
Elke onderneming krijgt toestemming om een bepaalde hoeveelheid uit te stoten, maar deze hoeveelheden zijn verhandelbaar. Stel dat elke onderneming even veel rechten krijgt, in figuur~\ref{fig:verhandelbareEmissierechten} komt dit overeen met een uitstoot van $OK = \frac{OQ_1 + OQ_2}{2}$. De prijs van de emissierechten zal dan op $MMB_{tv}$ komen te liggen.
\begin{figure}[htbp]
   \centering
   \includegraphics[scale=0.4]{Images/white.png}
   \caption{Verhandelbare emissierechten (slide 35)}
   \label{fig:verhandelbareEmissierechten}
\end{figure}
Onderneming 1 zal dan dus $Q_1K$ verkopen aan de makrtprijs, en zal winst $MEF$ maken (het gebied onder $MK$ zijn de kosten voor het terugdringen v/d vervuiling). Onderneming 2 daarentegen zal $KQ_2$ kopen en door deze aankoop het gebied $GNE$ winnen.

Het pareto optimum zal dus bereikt worden, mits de totale vervuiling in het optimum gekend is, en de totaal toegelaten uitstoot hieraan gelijk is. $MK$ is namelijk overal gelijk aan $p$.

Waarom het is dat de prijs steeds gelijk is aan $MMB_{tv}$ zie je in figuur~\ref{}.

\begin{figure}[htbp]
   \centering
   \includegraphics[scale=0.4]{Images/white.png}
   \caption{Prijsvorming emissierechten (slide 40)}
   \label{fig:prijsvormingEmissierechten}
\end{figure}

Hoe de emissierechten verdeeld wordt over de ondernemingen maakt ook niet uit, dit evenwicht zal steeds tot stand komen.

\subsection{Publieke Voorziening van Private Goederen}
Soms zal de overheid subsiedies voor private goederen zoals onderwijs en gezondheidszorg geven, of zelfs de productie voor zijn rekening nemen. Mogelijke motieven:
\begin{itemize}
   \item Externe effecten mee in rekening brengen op de markt
   \item Verdeling
   \item Verdienstengoederen: voorkeuren van consumenten be\"invloeden (vb subsidies voor opera om er mensen heen te lokken).
\end{itemize}
