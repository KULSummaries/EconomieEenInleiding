\section{De Onderneming als Organisatie}
De volgende hoofdstukken gaan over hoeveel de producenten gaan produceren en tegen welke prijs ze die productie gaan verkopen. In dit hoofdstuk: aan welke voorwaarden moet voldaan zijn om een maximale winst te hebben.

\subsection{Waarom Ondernemingen?}
eenmanszaak, BVBA, NV, blablabla...

Waarom zijn er nu ondernemingen en gaat niet alles via individuele spelers die hun output op de markt verkopen? Een individu neemt dan een taak op zich die in ons systeem een stap in het proces van een onderneming is.

Dit komt door de transactiekosten dat die werkwijze met zich meebrengt. Op termijn zullen er contracten tussen individuen ontstaan, maar deze contracten kunnen onmogelijk op alle eventualiteiten inspelen $\rightarrow$ mensen gaan zich groeperen, autoriteit aan \'e\'en individu afstaan om deze contracten dan te internaliseren binnen het bedrijf dat ontstaat.

\subsection{Winstmaximalisatie}
Ondernemingen zijn voornamelijk ge\"interesseerd in het maximaliseren van hun winst. Deze wordt beschreven met volgende functie:
\begin{equation}
	W(q) = TO(q) - TK(q)
\end{equation}

Deze bestaat dus uit de opbrengsten min de kosten. Deze kosten zijn de economische kosten, dus ook zaken zoals opportuniteitskosten worden in rekening gebracht. Aangezien $W$ in functie staat van $q$, zullen we de $q$ zoeken die $W$ maximaliseert. 

\subsubsection{Ontvangstenfuncties}
Volgende functies zijn relevant aan de opbrengstenzijde:
\begin{subequations}
\begin{align}
	TO(q) &= q \times q(p)\\
    GO(q) &= \frac{TO(q)}{q} = \frac{q \times p(q)}{q} = p(q)\\
    MO(q) &= \frac{\Delta TO(q)}{\Delta q} \rightarrow MO(q) = \frac{\mathrm{d} TO(q)}{\mathrm{d}q}
\end{align}
\end{subequations}
$p(q)$ is daarbij de inverse vraagfunctie. Merk op dat wanneer $q$ veranderd, $TO$ twee effecten zal ondergaan(figuur~\ref{fig:GOenTO}): een effect van de veranderde hoeveelheid en een verandering in de andere richting van de prijs. Het totale effect zal van de $\varepsilon_{V}^p$ afhangen. Als je naar het middelpunt beweegt zal $TO$ stijgen, als je weg beweegt dalen. Dit vertaalt zich ook in het snijden van $MO$ met de x-as in het punt waar die $q$ ook bij het punt hoort waar $|\varepsilon_{V}^p| = 1$.

\begin{figure}[htbp]
	\centering
	\includegraphics[scale=0.4]{Images/white.png}
	\caption{Inkomens- en substitutie-effect grafisch weergegeven}
	\label{fig:GOenTO}
\end{figure}

$GO$ en $MO$ kunnen we ook afleiden als we $TO$ als gegeven hebben. In elke $q$ is de waarde van $MO$ gelijk aan de afgeleide van $TO$ in dat punt. In elke $q$ is $GO$ gelijk aan $TO$ in dat punt gedeeld door die $q$.


\subsubsection{Kostenfuncties}
=verband tussen produciehoeveelheid en de kosten om die hoeveelheid te kunnen produceren. Analoog aan de opbrengstenfuncties hebben we nu $TK$, $GK$ en $MK$, waarbij $MK$ altijd positief moet zijn aangezien er altijd meer geld nodig zal zijn voor extra inputs. We gaan er voorlopig van uit dat $TK$ niet lineair is. Waarom juist zien we in hoofdstuk~\ref{sec:productieEnKosten}.

Het gevolg van die niet-lineariteit is dat zowel $MK$ en $GK$ eerst dalen en daarna terug stijgen. $MK$ snijdt $GK$ ook in zijn laagste punt en ligt voor dat punt onder $GK$. Ook hier kunnen $MK$ en $GK$ afgeleid worden uit de curve van $TO$ op een analoge manier als voorheen.


\subsubsection{Optimale Productie}
Er zijn twee voorwaarden om in een punt van optimale productie te zijn:
\begin{enumerate}
	\item Zoek de $q^*$ met de hoogste winst, waarbij $q^* > 0$
    \item \textbf{sluitingsregel}: ga na of de winst in deze $q^*$ ook effectief positief is
\end{enumerate}

Het zoeken naar een optimale $q^*$ kunnen we op twee manieren doen: kijken waar $TW$ het grootst is, met $ TW(q) = TO(q) - TK(q)$. Een andere, meer gebruikte manier is zoeken naar het punt waar $MW(q) = 0$ met $MW(q) = MO(q) - MK(q)$. Bij deze laatste methode moet je wel zorgen dat:
\begin{equation}
	\left \{\begin{array}{l l}
    MO(q) > MK(q) \text{ voor } q < q^*\\
    MO(q) < MK(q) \text{ voor } q > q^*
  \end{array} \right.
\end{equation}

We weten nu dat in $q^*$ de winst maximaal is voor alle positieve productieniveaus, maar het kan nog altijd zijn dat je daar verlies maakt ($q^*$ is de prod. met het minimale verlies).

$\rightarrow$ controleren of $TW(q)$ in $q^*$ positief is (=\textbf{sluitingsregel})

\subsection{Kritiek op Model van de Winstmaximaliserende Onderneming}
Twijfels rond model van winstmaximalisatie:
\begin{description}
	\item[Gedragstheorie\"en]: zijn mensen wel zo rationeel om winstmaximaliserende keuzes te maken?
	\item[Managementtheorie\"en]: eigenaars zijn aandeelhouders en willen dus een zo hoog mogelijke winst, maar staan niet in voor dagdagelijkse leiding $\Rightarrow$ misschien heeft andere objectieven.
\end{description}

Voor het eerste punt van kritiek zijn er vele empirische studies (bvb. Levitt-Feldman) die aantonen dat $MO$ en $MK$ elkaar toch zullen benaderen. Het tweede punt is ook bekend onder het \textbf{principaal-agent probleem}. Dit stelt dat het management eerder prestige nastreeft, wat verbonden is met een grote omzet. Een grote omzet nastreven zal echter niet de maximale winst garanderen. $\Rightarrow$ beperkte controle \& prestige = probleem.

Oplossingen die hiervoor kunnen dienen:
\begin{itemize}
	\item interne controlemechanismen:
	\begin{itemize}
		\item \textbf{rechtstreekse controle} door de raad van bestuur doormiddel van (in)formele regels
		\item \textbf{onrechtstreekse controle} door het toedienen van de juiste prikkels aan het management (bonussen, aandelenopties,...)
	\end{itemize}
	\item externe controlemechanismen: 
	\begin{itemize}
		\item slechte prestatie $\Rightarrow$ lage koers $\Rightarrow$ dreiging van overname
	\end{itemize}
\end{itemize}