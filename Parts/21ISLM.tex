\section{Het IS-LM Model}
We nemen het Keynesiaanse model als basis, maar we gaan er nu van uit dat de investeringen niet meer exogeen gegeven zijn. Ze hangen echter af van $i$, wat zal zorgen voor een wisselwerking tussen de re\"ele en de geldmarkt. We krijgen dus een investeringsfunctie:
\begin{equation}
  I = I_0 - hi
\end{equation}

Deze $i$ is dus ook endogeen, want deze wordt gezet op de geldmarkt.

Het effect van deze wisselwerking illustreren we aan de hand van een voorbeeld: een expansief budgettair beleid. Stel dat $G\uparrow$, dan zal in eerste instantie $Q\uparrow$. Dit zorgt ervoor dat de transactievraag naar geld toeneemt, $M$ zal moeten stijgen (denk ook aan Fisher). Als $M^v\uparrow$, zal $i\uparrow$. Het gevolg van deze hogere interestvoet is dat $I$ zal afnemen. De oorspronkelijk gestegen $Q$ zal terug wat afnemen. Dit effect noemt men \textbf{crowding out}.

\subsection{De IS-curve}
Neem het model van vorig hoofdstuk, maar dan zonder het buitenland. Indien we onze nieuwe vergelijking voor $I$ hier in invoegen, bekomen we een iets correcter model, met de nieuwe elementen $h$ en $i$:
\begin{align}
  Q &= \frac{C_0 + I_0 + G - hi}{1-c(1-t)} \\
  \Leftrightarrow i &= \frac{C_0 + I_0 + G}{h} - \frac{1-c(1-t)}{h}Q
\end{align}

We zien dat we dit nieuwe model anders kunnen schrijven. Deze laatste manier van schrijven, waarbij $i$ in functie van $Q$ staat, noemen we de IS-curve. Deze curve geeft dus weer voor welke combinaties van $i$ en $Q$ de re\"ele markt in evenwicht is.

\subsection{De LM-curve}
Net zoals de IS-curve de evenwichten op de re\"ele markt weergaf, kunnen we dit ook doen voor de geldmarkt. Ook hier zoeken we $i$ in functie van $Q$:
\begin{align}
  M^A &= P(L_0 + \beta Q - \delta i) \\
  \Leftrightarrow i &= \frac{L_0 + \beta Q}{\delta} - \frac{1}{\delta} \left[\frac{\overline{M^A}}{P}\right]
\end{align}

\subsection{Macro-Economisch Evenwicht op Goederen- en Geldmarkt}
Het is eenvoudig in te zien dat we beide vergelijkingen kunnen samenbrengen in \'e\'en model. Aangezien ze voorheen allebei evenwichten toonden, maar $i$ niet kenden, zullen we het evenwicht op deze manier wel kunnen vinden. Om het volledige model op te stellen moet je het buitenland nog toevoegen. Tot slot nog twee termen:

\begin{description}
  \item[Liquidity trap]: In werkelijkheid zal de LM-curve wellicht geen rechte zijn, maar eerder zeer vlak worden als $i$ laag staat. Dit wil zeggen dat als je ``in de liquidity trap'' zit, een monetair beleid voeren geen grote impact zal hebben op $Q$.
  \item[Policy mix]: Zie voorbeelden waarom het belangrijk is monetair en budgettair beleid op elkaar af te stemmen. (Bill Clinton, Duitse eenmaking en het huidig beleid)
\end{description}
