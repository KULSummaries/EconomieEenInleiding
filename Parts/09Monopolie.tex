\section{Monopolie}
=\'e\'en aanbieder aanwezig zonder goede substituten voor zijn product. Monopolist wordt dus als enige geconfronteerd met de marktvraag. $\Rightarrow$ monopolist heeft marktmacht en is \textbf{prijszetter} (binnen bepaalde marges). Bijgevolg is de ondernemingsvraag = marktvraag. Monopolist zal dus moeten kiezen, hogere $p$ bij lagere $q$ of omgekeerd. De keuze hangt af van $\varepsilon^V_p$.

\subsection{Winstmaximalisatie bij Monopolie}
\subsection{Prijsdiscriminatie}
\subsection{Oorzaken van Monopolie}
\subsection{Welvaartsanalyse van Monopolie}
